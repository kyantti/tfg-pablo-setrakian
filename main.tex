\documentclass[a4paper,12pt,openany ,oneside]{book}  

%Paquetes
\usepackage[utf8]{inputenc}
\usepackage[spanish]{babel}
\usepackage{graphicx} 
\usepackage{tikz,color}
\usepackage{pgf-pie} 
\usepackage{array}
\usepackage{wrapfig}   
\usepackage[scriptsize]{caption} 
\usepackage{amsmath} 
\usepackage{amsthm}  
\usepackage{float}   
\usepackage{lastpage}
\usepackage{pdfpages}
\usepackage{url}
\usepackage{lscape}
\usepackage{csvsimple}
% Package rotating -> Para poder girar las tablas y dibujarlas a lo largo
% del folio en vez de a lo ancho.
\usepackage{rotating}
% Packages multicol y multirow, para manejar tablas de filas y columnas múltiples.
\usepackage{multicol}
\usepackage{multirow}
%Para codigo maquina
\usepackage{listings}
%Para hypervínculos en el indice
\usepackage{hyperref}
\usepackage{pdfpages}
\usepackage{lscape}
\usepackage{eurosym}
\usepackage{adjustbox}
\usepackage{algpseudocode}
\usepackage{graphicx}
\usepackage{luatex85}

\usepackage{appendix}

\usepackage{epstopdf} 
\usepackage{subcaption}
%Renombramientos

\renewcommand{\contentsname}{Índice de contenido}
\renewcommand{\chaptername}{Capítulo}            
\renewcommand{\bibname}{Referencias} 
\renewcommand{\listfigurename}{Índice de figuras}
\renewcommand{\listtablename}{Índice de tablas}
\renewcommand{\figurename}{Figura}
\renewcommand{\tablename}{Tabla}
\addto\captionsspanish{\renewcommand{\appendixname}{Anexo}}
%\renewcommand{\appendixname}{Anexo}
\renewcommand{\appendixtocname}{Anexos}
\renewcommand{\appendixpagename}{Anexos}
\renewcommand{\lstlistingname}{Listado}

\newcolumntype{C}[1]{>{\centering\arraybackslash}m{#1}}

\providecommand{\keywords}[1]{\textbf{\textit{Keywords---}} #1}

%\setcounter{secnumdepth}{4}
%\setcounter{tocdepth}{4}

%Configuración páginas
%\include{config/paginas}

\begin{document}

\pagestyle{empty}



\begin{center}
	\newcommand{\HRule}{\rule{\linewidth}{0.7mm}}
	\begin{center}
            \includegraphics[width=\columnwidth, keepaspectratio]{images/logo.png}\\
        \end{center}

	\vspace*{2cm}
	\textsc{\large Universidad de Extremadura}\\[0.5cm]
	\textsc{\large Centro Universitario de Mérida}\\[1.3cm]
	\textsc{\LARGE Grado en Ingeniería Informática en Tecnologías de la Información}\\[1.3cm]
	\textsc{\huge Trabajo Fin de Grado}\\[2cm]
	\textsc{\LARGE Detección de micotoxinas en higos frescos mediante técnicas de inteligencia artificial }\\[1.5 cm]
	
	\textsc{\large Pablo Setrakian Bearzotti}
	\vspace{2cm} 
	\vfill
	Mérida, Junio de 2024
\end{center}

\cleardoublepage
\pagestyle{empty}

\begin{center}
	\newcommand{\HRule}{\rule{\linewidth}{0.7mm}}
	\begin{center}
            \includegraphics[width=\columnwidth, keepaspectratio]{images/logo.png}\\
        \end{center}
	
	\vspace*{2cm}
	\textsc{\large Universidad de Extremadura}\\[0.5cm]
	\textsc{\large Centro Universitario de Mérida}\\[1.3cm]
	\textsc{\LARGE Grado en Ingeniería Informática en Tecnologías de la Información}\\[1.3cm]
	\textsc{\huge Trabajo Fin de Grado}\\[2 cm]
	\textsc{\LARGE Optimización de problema de riego eficiente en higuera mediante inteligencia artificial}\\[1.5 cm]

\end{center}

\begin{flushleft}
	\begin{tabular}{ll}
		% Nombre del alumno.
		\large{\textbf{Autor:}}	&
		\large{Pablo Setrakian Bearzotti} \\
		\large{\textbf{Fdo:}} & \\
		\\
		\\
		% Nombre del director/tutor del proyecto.
		\large{\textbf{Director:}}	&
		\large{Francisco Chávez de la O} \\
		\large{\textbf{Fdo:}} & \\
	\end{tabular}
\end{flushleft}

\cleardoublepage

\pagestyle{plain}
\pagenumbering{Roman} % para comenzar la numeración de paginas en números romanos

\chapter*{Agradecimientos} 
\addcontentsline{toc}{chapter}{Agradecimientos} % si queremos que aparezcan en el índice

En primer lugar, quiero agradecer a mis padres por animarme a estudiar ingeniería informática y por su apoyo incondicional durante toda mi formación académica.

A mi profesor Francisco Chávez de la O por sus clases, las cuales me hicieron empezar a interesarme por la inteligencia artificial y por su propuesta para participar en este proyecto.

\cleardoublepage

\renewcommand{\keywords}{{\itshape \bfseries Palabras clave - }}
\chapter*{Resumen} 
%\addcontentsline{toc}{chapter}{Resumen} % si queremos que aparezca en el índice

La seguridad alimentaria representa uno de los desafíos más críticos de la actualidad, especialmente en el contexto de la detección temprana de contaminantes que pueden representar riesgos significativos para la salud humana. Las micotoxinas, sustancias tóxicas producidas por hongos como el Aspergillus flavus, constituyen una amenaza considerable en la cadena alimentaria, siendo clasificadas por la Agencia Internacional para la Investigación del Cáncer como sustancias carcinógenas del grupo 1.

\vspace{5mm}

La detección tradicional de micotoxinas en productos agrícolas requiere métodos invasivos que implican la destrucción de las muestras, lo que resulta en pérdidas económicas significativas y limitaciones en el control de calidad durante el proceso productivo. En este contexto, el desarrollo de técnicas no invasivas para la detección temprana de contaminación por aflatoxinas se presenta como una necesidad imperante para la industria agroalimentaria.

\vspace{5mm}

Este Trabajo Fin de Grado se centra en el desarrollo de un sistema de inteligencia artificial capaz de detectar la contaminación por micotoxinas en higos frescos mediante el análisis de imágenes hiperespectrales y técnicas de inteligencia artificial. El proyecto aborda específicamente la clasificación de estados de enfermedad en frutos de higuera, utilizando un algoritmo genético junto a una red neuronal para identificar la presencia de Aspergillus flavus en diferentes concentraciones de contaminación.

\vspace{5mm}

La metodología propuesta se estructura en múltiples fases, comenzando con la localización y segmentación de higos individuales mediante detección RGB, seguida de la selección de bandas espectrales más informativas utilizando un algoritmo genético, y culminando con el procesamiento de parches espectrales completos mediante transformadas wavelet. Esta aproximación multimodal permite aprovechar tanto la información espectral específica como las características espaciales de las imágenes hiperespectrales.

\vspace{5mm}

Los resultados preliminares demuestran la viabilidad del enfoque propuesto para la detección no invasiva de contaminación por micotoxinas, estableciendo un marco de trabajo que contribuirá significativamente a mejorar los estándares de seguridad alimentaria en la producción de higos, especialmente relevante para regiones productoras como Extremadura, que representa el 55.5\% de la producción nacional española.

\vspace{5mm}

\keywords{inteligencia artificial, imágenes hiperespectrales, aprendizaje automático, visión por computador, algoritmo genético}

\cleardoublepage

\renewcommand{\keywords}{{\itshape \bfseries Keywords - }}
\chapter*{Abstract} 

Artificial intelligence is no longer just a futuristic promise, but a reality that is redefining the way we interact with the world and conduct business. From healthcare to the manufacturing industry, AI has demonstrated its ability to radically transform diverse fields, unleashing a series of changes that are revolutionising today. Artificial intelligence will enable the automation of a wide range of complex processes that currently still require human intervention.  

\vspace{5mm}

Artificial intelligence which, through methods, techniques and algorithms, provides computers the ability of identifying patterns in massive data and make predictions (predictive analytics). This learning allows computers to carry out specific tasks autonomously, i.e. without the need to be programmed.

\vspace{5mm}

The constant innovation and evolution of the emerging technologies, paired with the use by the vast majority of the world’s population, has led to organisations exploiting the power of the automatic learning methods to help them extract better quality information, increase productivity, reduce the costs and get more value form their data.

\vspace{5mm}

In an ever-evolving world where technology is redefining the way we live, agriculture is no exception. Artificial intelligence is emerging as a key tool to help farmers adapt to the effects of climate change and lessen its impact. By capitalising on artificial intelligence, farmers can make informed, data-driven decisions and anticipate the climate shifts.

\vspace{5mm}

This research deals with sustainable production systems in the cultivation of fig trees and the study of water needs in order to design irrigation strategies using a set of automatic learning techniques belonging to the world of artificial intelligence.

\vspace{5mm}

Lastly, once the results generated by the entire system created have been obtained, these will be analysed and the solutions to the problems presented will be identified. These will allow us to observe the degree of efficiency of the experiments developed and to know if they meet the proposed targets.


\keywords{Artificial Intelligence, Agriculture, Climate change}

\cleardoublepage

\tableofcontents

\addcontentsline{toc}{chapter}{Índice de figuras}
\listoffigures

\renewcommand{\listtablename}{Índice de tablas}
\renewcommand{\tablename}{Tabla}

\cleardoublepage
\addcontentsline{toc}{chapter}{Índice de tablas}
\listoftables
\cleardoublepage

%\pagestyle{fancy} 
\pagenumbering{arabic}

% CAPÍTULOS -----------------------------------
\chapter{Introducción}
\section{Introducción}
Las enfermedades y plagas en cultivos representan un desafío económico significativo para los sectores agrícola y alimentario a nivel mundial. Entre las amenazas más graves se encuentran las micotoxinas, sustancias producidas naturalmente por ciertos tipos de hongos bajo condiciones particulares de humedad y temperatura \cite{KHAN2024e28361}. La presencia de micotoxinas en alimentos constituye un problema serio tanto para la salud humana como animal. La Agencia Internacional para la Investigación del Cáncer (IARC) ha clasificado un grupo de aflatoxinas como sustancias carcinogénicas del grupo 1, siendo la vía común de exposición a micotoxinas la ingesta de alimentos contaminados \cite{iarc2012aflatoxins}.

\vspace{5mm}

El hongo \textit{Aspergillus flavus}, que prolifera a temperaturas entre 12°C y 27°C con 85\% de humedad, se multiplica en diversos alimentos incluyendo maíz, cacahuetes, arroz, frutos secos e higos \cite{10.1093/fqsafe/fyz040}. Aunque su presencia es típica de climas tropicales, también prolifera bajo ciertas condiciones de riego. El ciclo de crecimiento de la aflatoxina es de entre 3 y 5 días \cite{Ahmad2025}. Incluso si los higos van a ser secados, la introducción de higos infectados con aflatoxinas en el proceso puede provocar la contaminación de otros frutos. Por tanto, la detección de aflatoxinas en el producto fresco se considera crucial tanto para el consumo directo como para su procesamiento posterior. 

\vspace{5mm}

El cultivo de la higuera (\textit{Ficus carica} L.) tiene sus orígenes en la región de Caria en Asia, habiéndose extendido a otras áreas como la región mediterránea, África y América \cite{cabicompendium.24078}. En 2023 España fue actualmente el octavo mayor productor mundial, representando el 3.05\% de la producción global. La región de Extremadura, con 12,771 hectáreas cultivadas, representa el mayor productor en España, alcanzando el 55.5\% de la producción nacional \cite{esyrce2023}. El aumento en la productividad está vinculado a la adopción de técnicas innovadoras como fertilización, poda, tratamiento del suelo e irrigación. Sin embargo, los cambios en la humedad facilitan la propagación de la micotoxina \textit{Aspergillus flavus}, requiriendo investigación adicional para analizar y prevenir que higos infectados entren en la cadena alimentaria humana.

\section{Motivación}
La detección tradicional de aflatoxinas se realiza mediante métodos invasivos que requieren la destrucción de la muestra, o mediante inspección visual en etapas avanzadas de contaminación. Estos métodos presentan limitaciones significativas: son lentos, costosos, y no permiten el análisis en tiempo real durante el proceso productivo \cite{KHAN2022101678}. Además, los higos frescos son perecederos, tienen una vida útil limitada y son más sensibles al crecimiento microbiano que los higos secos, alterando la calidad del producto y representando un riesgo serio para la salud humana.

\vspace{5mm}

El uso de imágenes hiperespectrales (\emph{HSI}) combinado con técnicas de inteligencia artificial, particularmente el aprendizaje profundo, ofrece una alternativa prometedora. La tecnología \emph{HSI} mide la interacción de un amplio espectro de luz con un objeto determinado, adquiriendo cientos de bandas espectrales contiguas para cada píxel en una imagen. Esta capacidad proporciona información detallada sobre el objeto y revela diferencias sutiles en textura y composición química que no son detectables mediante métodos convencionales.

\vspace{5mm}

La necesidad de desarrollar métodos no invasivos y precisos para la detección temprana de contaminación por aflatoxinas en higos frescos es crítica para garantizar la seguridad alimentaria, reducir pérdidas económicas en la cadena de producción, y proteger la salud pública.

\section{Objetivo general}
Desarrollar un sistema de inteligencia artificial basado en el análisis de imágenes hiperespectrales para la detección temprana de contaminación por micotoxinas en higos frescos, utilizando técnicas de aprendizaje profundo y algoritmos genéticos para optimizar la selección de características espectrales relevantes. Asimismo, se emplearán métodos de extracción y transformación de información espectral a partir de regiones de interés en las imágenes, con el objetivo de generar representaciones avanzadas que faciliten la evaluación mediante modelos de redes neuronales profundas.

\section{Objetivos específicos}
\begin{itemize}
    \item Implementar un sistema de detección y segmentación automática de higos individuales en imágenes RGB mediante técnicas de visión por computador, generando máscaras y anotaciones para su posterior extracción de datos hiperespectrales.
    \item Desarrollar e implementar un algoritmo genético para la selección óptima de las tres bandas espectrales más informativas del cubo hiperespectral, reduciendo la dimensionalidad de los datos mientras se mantiene la capacidad discriminativa.
    \item Diseñar y entrenar modelos de redes neuronales profundas capaces de clasificar el estado de contaminación de los higos basándose en las bandas espectrales seleccionadas, evaluando diferentes arquitecturas y configuraciones.
    \item Implementar métodos de extracción y transformación de información espectral a partir de regiones de interés en las imágenes hiperespectrales, generando representaciones optimizadas que mejoren la precisión y eficiencia de los modelos de clasificación.
\end{itemize}

\newpage
\section{Planificación}
El desarrollo del proyecto se estructura en las siguientes fases principales, diseñadas para abordar progresivamente los desafíos técnicos y científicos:

\begin{enumerate}
    \item \textbf{Preparación y adquisición de datos:} Recolección del dataset de imágenes hiperespectrales incluyendo muestras contaminadas con diferentes niveles de micotoxinas y muestras de control no contaminadas, capturadas durante un período de dos semanas para garantizar diversidad y robustez.
    \item \textbf{Detección y Segmentación de Figuras:} Desarrollo del sistema de detección automática de higos individuales mediante modelos de detección de figuras y segmentación aplicados a versiones \emph{RGB} de las imágenes hiperespectrales, generando máscaras y anotaciones.
    \item \textbf{Selección de bandas:} Implementación de un algoritmo genético para identificar las tres bandas espectrales más informativas del cubo hiperespectral, construyendo imágenes reducidas para el entrenamiento de redes neuronales. La selección se basará en la capacidad de las bandas para discriminar entre higos contaminados y no contaminados.
    \item \textbf{Extracción y transformación de información espectral:} Implementación de técnicas avanzadas para extraer y transformar la información espectral de las regiones de interés, generando representaciones en formato imagen para su uso en modelos de redes neuronales profundas.
    \item \textbf{Comparación de resultados:} Evaluación comparativa de los modelos de redes neuronales entrenados con las bandas seleccionadas mediante el algoritmo genético y las representaciones generadas a partir de la extracción y transformación de información espectral, analizando métricas de rendimiento, eficiencia computacional y consumo energético.
\end{enumerate}

\section{Organización del documento}
El presente documento se estructura en los siguientes capítulos para presentar de manera sistemática el desarrollo y resultados del proyecto:

\begin{itemize}
    \item \textbf{Capítulo 2. Marco teórico y estado del arte:} Presenta los fundamentos teóricos de las imágenes hiperespectrales, técnicas de aprendizaje profundo aplicadas a la agricultura de precisión, y una revisión exhaustiva de trabajos relacionados con la detección de aflatoxinas mediante métodos no invasivos.
    \item \textbf{Capítulo 3. Desarrollo:} Detalla la metodología implementada en cada fase del proyecto, incluyendo la arquitectura del sistema de detección y segmentación, el diseño del algoritmo genético, y la implementación de los modelos de redes neuronales profundas.
    \item \textbf{Capítulo 4. Resultados:} Presenta los resultados experimentales obtenidos en cada fase, incluyendo métricas de rendimiento, análisis comparativo entre diferentes aproximaciones, y evaluación del impacto computacional y energético de los modelos.
    \item \textbf{Capítulo 5. Conclusiones y trabajo futuro:} Resume las contribuciones principales del proyecto, discute las limitaciones encontradas, y propone líneas de investigación futuras para mejorar y extender el sistema desarrollado.
\end{itemize}

\chapter{Estado del Arte}

\section{Introducción}

La agricultura de precisión ha experimentado una transformación significativa en las últimas décadas, impulsada por el desarrollo de tecnologías emergentes que permiten el monitoreo y análisis automatizado de cultivos \cite{CISTERNAS2020105626, KHAN2022101678}. En este contexto, las tecnologías de imagenología avanzada, particularmente las imágenes hiperespectrales (HSI) y RGB, han emergido como herramientas clave para la detección temprana de contaminantes y patógenos en productos agrícolas. La integración de estas tecnologías con algoritmos de machine learning ha abierto nuevas posibilidades para el desarrollo de sistemas de detección no invasivos, precisos y eficientes \cite{KHAN2022101678, jimaging5050052, agriengineering6040225}. 
    
Este capítulo presenta una revisión sistemática del estado del arte en tecnologías de detección de contaminantes en productos agrícolas, centrándose en la aplicación de imágenes hiperespectrales y RGB en combinación con técnicas de aprendizaje automático. El análisis ofrece el marco teórico necesario para comprender las contribuciones de este proyecto y su relevancia dentro del panorama científico actual.

\section{HSI en Agricultura de Precisión}

La imagenología hiperespectral constituye un avance notable en el análisis remoto, ya que permite capturar información extremadamente detallada a través de cientos de bandas espectrales contiguas por cada píxel \cite{article}. Este enfoque mide cómo un amplio rango del espectro electromagnético interactúa con un objeto, proporcionando información sobre su composición química y revelando variaciones sutiles que los métodos convencionales no pueden detectar \cite{WIEME2022156}. La capacidad de HSI para capturar características a través de múltiples bandas permite crear firmas únicas que representan cómo diferentes materiales responden a cada longitud de onda. Según la literatura científica, cuanto mayor es el número de bandas, más detalladas son las características que pueden ser identificadas, aunque no todas las bandas incluyen información relevante para mejorar la precisión de detección \cite{HONG201935}.

\vspace{5mm}

Los sistemas HSI típicamente operan en diferentes rangos espectrales, incluyendo el visible (VIS: 400-700 nm), infrarrojo cercano (NIR: 700-1000 nm), infrarrojo de onda corta (SWIR: 1000-2500 nm), y otros rangos especializados. Esta versatilidad espectral permite identificar características específicas de materiales y cambios químicos imperceptibles para el ojo humano o sistemas de imagen convencionales.

\vspace{5mm}

La aplicación de HSI en la detección de contaminantes agrícolas ha mostrado resultados prometedores en numerosos estudios. Estas investigaciones han abordado la identificación de infecciones fúngicas, micotoxinas y otros patógenos que comprometen la calidad y la seguridad de los productos alimentarios. En el contexto específico de detección de aflatoxinas, varios estudios han explorado el potencial de HSI para la identificación temprana de contaminación. La investigación ha demostrado que las aflatoxinas, particularmente la Aflatoxina B1 producida por Aspergillus flavus, pueden ser detectadas utilizando análisis espectral no invasivo. Estudios recientes han aplicado HSI con cámaras VNIR (400-1000 nm) y SWIR (1000-2500 nm) en diversos cultivos, logrando resultados prometedores en la identificación de muestras contaminadas frente a controles sanos.

\vspace{5mm}

La distribución superficial de las aflatoxinas representa una ventaja particular para el análisis mediante HSI, ya que permite detectar cambios químicos y estructurales en las capas exteriores de los productos agrícolas. Esta característica facilita la implementación de sistemas de detección que no requieren la destrucción de las muestras, manteniendo la integridad del producto para su comercialización.

\vspace{5mm}

A pesar de las ventajas evidentes de HSI, existen desafíos significativos asociados con su implementación práctica. El principal desafío radica en la alta dimensionalidad de los datos hiperespectrales, que puede representar un obstáculo considerable para los algoritmos de clasificación tradicionales \cite{HONG201935}. Los cubos hiperespectrales generan volúmenes masivos de datos complejos que requieren técnicas especializadas de procesamiento y análisis. La gestión de la dimensionalidad espectral requiere estrategias de selección de características y reducción dimensional para identificar las bandas espectrales más informativas. La eliminación de bandas espectrales redundantes no solo facilita el análisis computacional, sino que también puede mejorar la precisión de clasificación al reducir el ruido y la información irrelevante. Adicionalmente, las condiciones de adquisición de imágenes hiperespectrales requieren un control cuidadoso de factores ambientales como la iluminación, temperatura y humedad, que pueden afectar la calidad y consistencia de los datos espectrales. La calibración y normalización de los datos espectrales son procedimientos críticos para garantizar la reproducibilidad y confiabilidad de los resultados.

\section{Imágenes RGB en Agricultura de Precisión}

En el ámbito agrícola, las imágenes RGB siguen siendo la tecnología más extendida \cite{FERENTINOS2018311}. Aunque limitadas en comparación con HSI, destacan por su bajo costo, simplicidad de uso y rapidez en el procesamiento, cualidades que han facilitado su implementación a gran escala. Los sistemas RGB capturan únicamente tres bandas espectrales correspondientes a los colores primarios, generando representaciones visuales similares a la percepción humana. A pesar de su simplicidad, esta información resulta suficiente para detectar variaciones morfológicas y de color asociadas con infecciones fúngicas u otros contaminantes.

\vspace{5mm}

La literatura científica documenta numerosas aplicaciones exitosas de imágenes RGB en la detección de contaminantes agrícolas. Estos sistemas han demostrado eficacia particular en la identificación de cambios visuales asociados con infecciones fúngicas, decoloración y alteraciones morfológicas que preceden o acompañan la contaminación por micotoxinas. En el contexto de detección de aflatoxinas, algunos estudios han explorado el uso de imágenes RGB para identificar cambios visuales en productos contaminados. Aunque la información espectral limitada de RGB puede restringir la detección de cambios químicos sutiles, la tecnología ha mostrado utilidad en la identificación de síntomas visuales avanzados de contaminación fúngica.

\vspace{5mm}

Las aplicaciones RGB se han extendido también a sistemas de clasificación automatizada para el control de calidad en líneas de producción, donde la velocidad de procesamiento y la simplicidad del sistema son factores críticos. Estos sistemas pueden proporcionar una primera línea de defensa en la detección de productos visiblemente afectados.

\vspace{5mm}

Las principales limitaciones de los sistemas RGB radican en su capacidad limitada para detectar cambios químicos sutiles que no se manifiestan visualmente. La contaminación por micotoxinas puede ocurrir sin síntomas visuales evidentes en las etapas tempranas, limitando la efectividad de los sistemas RGB para la detección precoz. Adicionalmente, los sistemas RGB son susceptibles a variaciones en las condiciones de iluminación y pueden requerir normalización cuidadosa para mantener la consistencia en diferentes entornos. La dependencia de características visuales también puede resultar en falsos positivos cuando se presentan variaciones naturales en color o textura que no están relacionadas con contaminación.

\section{ML en Imagenología Agrícola}

La aplicación de Deep Learning (DL) en el procesamiento de imágenes hiperespectrales ha revolucionado las capacidades de análisis y clasificación en agricultura de precisión \cite{PAOLETTI2019279}. Como subconjunto del machine learning, el deep learning utiliza redes neuronales profundas que consisten en múltiples capas interconectadas de neuronas artificiales capaces de aprender representaciones de alto nivel a partir de datos de entrada.

\vspace{5mm}

La implementación de DL para el procesamiento y análisis de imágenes hiperespectrales fue inicialmente descrita en investigaciones pioneras que propusieron enfoques de clasificación utilizando información espacialmente dominante \cite{chen2014deep}. Desde entonces, un gran número de estudios han reflejado el interés creciente de la comunidad científica en esta área de investigación.

\vspace{5mm}

Las redes neuronales convolucionales tridimensionales han emergido como una arquitectura particularmente efectiva para el procesamiento de cubos hiperespectrales \cite{Zhong}. Estas redes pueden capturar simultáneamente características espaciales y espectrales, aprovechando la naturaleza tridimensional inherente de los datos hiperespectrales. Las CNN 3D operan mediante la aplicación de filtros convolucionales tridimensionales que se desplazan a través de las dimensiones espaciales (x, y) y espectral (z) del cubo hiperespectral. Esta capacidad permite la extracción de características que consideran tanto la variabilidad espacial local como las relaciones espectrales entre bandas adyacentes.

\vspace{5mm}

Una alternativa prometedora al uso directo de redes 3D consiste en la aplicación de transformadas matemáticas al espectro hiperespectral antes del procesamiento con redes neuronales \cite{agriengineering6040225}. Las transformadas wavelet han mostrado particular eficacia en este contexto, permitiendo la descomposición del espectro en componentes de frecuencia que pueden ser procesados mediante arquitecturas de red más simples. El enfoque basado en transformadas wavelet ofrece ventajas computacionales significativas, ya que permite la conversión de firmas espectrales unidimensionales en representaciones bidimensionales que pueden ser procesadas eficientemente mediante CNN 2D convencionales. Esta metodología puede mantener la información espectral crítica mientras reduce la complejidad computacional del procesamiento.

\vspace{5mm}

Las redes neuronales convolucionales bidimensionales (CNN 2D) representan el estándar establecido para el procesamiento de imágenes RGB en aplicaciones agrícolas \cite{FERENTINOS2018311}. Estas arquitecturas han demostrado eficacia excepcional en tareas de clasificación, detección de objetos y segmentación semántica aplicadas a productos agrícolas. Las CNN 2D operan mediante la aplicación de filtros convolucionales que capturan características espaciales locales en las imágenes RGB. La jerarquía de capas permite la extracción progresiva de características, desde detectores de bordes y texturas en capas tempranas hasta representaciones semánticas complejas en capas profundas.


\vspace{5mm}

El desarrollo de arquitecturas especializadas para detección de objetos y segmentación semántica ha facilitado la implementación de sistemas automatizados de análisis agrícola. Arquitecturas como YOLO (You Only Look Once), R-CNN y sus variantes han demostrado eficacia en la detección y localización automática de productos agrícolas en imágenes RGB. Para tareas de segmentación semántica, arquitecturas como U-Net, SegNet y DeepLab han mostrado resultados prometedores en la delimitación precisa de regiones de interés en imágenes agrícolas. Estas capacidades son fundamentales para el análisis posterior de características específicas de productos individuales.

\vspace{5mm}

La integración de información RGB e hiperespectral representa una frontera emergente en el análisis agrícola automatizado \cite{jimaging5050052}. Los enfoques híbridos pueden aprovechar las ventajas complementarias de ambas modalidades: la simplicidad y velocidad del RGB para detección y localización, y la riqueza espectral de HSI para análisis químico detallado. Estos sistemas multimodales típicamente implementan arquitecturas de procesamiento en cascada, donde la información RGB se utiliza para la detección inicial y segmentación de productos, seguida por análisis hiperespectral detallado de las regiones de interés identificadas. Esta estrategia puede optimizar tanto la eficiencia computacional como la precisión de detección.

\vspace{5mm}

Las técnicas de fusión de características permiten la combinación sistemática de información extraída de diferentes modalidades de imagen. Estos enfoques pueden implementarse a diferentes niveles del pipeline de procesamiento: fusión temprana (combinación de datos raw), fusión intermedia (combinación de características extraídas) o fusión tardía (combinación de decisiones de clasificadores independientes). La fusión efectiva de características RGB e hiperespectrales requiere consideración cuidadosa de las diferencias en resolución espacial, rango dinámico y características estadísticas entre las modalidades. Técnicas de normalización y alineamiento espacial son críticas para el éxito de estos enfoques.

\vspace{5mm}

El análisis de la literatura revela fortalezas y limitaciones distintas en los enfoques actuales para la detección de contaminantes agrícolas. Los sistemas basados en HSI ofrecen capacidades superiores de detección química pero requieren recursos computacionales significativos y equipos especializados costosos \cite{KHAN2022101678}. Los sistemas RGB proporcionan soluciones más accesibles y eficientes pero con capacidades limitadas de detección temprana.

\vspace{5mm}

Las arquitecturas de deep learning han demostrado capacidades excepcionales en ambas modalidades, pero su implementación efectiva requiere datasets grandes y representativos que pueden ser costosos y tiempo-intensivos de generar. La transferibilidad de modelos entre diferentes cultivos, condiciones ambientales y sistemas de adquisición permanece como un desafío significativo.

Las oportunidades de innovación identificadas incluyen el desarrollo de arquitecturas híbridas que combinen eficientemente información RGB e hiperespectral, la implementación de técnicas de selección inteligente de características espectrales y el desarrollo de sistemas adaptativos que puedan operar efectivamente en condiciones variables de campo \cite{WIEME2022156}. La integración de técnicas de optimización evolutiva, como algoritmos genéticos, para la selección automática de características espectrales representa una dirección prometedora para mejorar tanto la eficiencia como la precisión de los sistemas de detección. Estos enfoques pueden automatizar el proceso de identificación de bandas espectrales óptimas para aplicaciones específicas.
\chapter{Desarrollo}

\section{Introducción}

El desarrollo del proyecto se ha estructurado en múltiples fases secuenciales, cada una diseñada para abordar aspectos específicos del proceso de análisis hiperespectral aplicado a la detección de aflatoxinas en higos frescos. La metodología desarrollada implementa técnicas de visión por computador de última generación combinadas con procesamiento especializado de datos hiperespectrales para crear un sistema automatizado de análisis de muestras.

\section{Entorno de Desarrollo}

El proyecto se ha implementado utilizando un entorno de desarrollo adaptado para procesamiento de imágenes hiperespectrales y ejecución de modelos de aprendizaje profundo. A continuación se detallan los aspectos fundamentales de la infraestructura técnica utilizada.

\subsection{Gestión de Entornos y Dependencias}

El proyecto se desarrolló utilizando el lenguaje de programación \emph{Python} \cite{van1995python} en su versión 3.13 con soporte para \emph{Cython} \cite{behnel2011cython}, lo que proporcionó ventajas significativas en términos de rendimiento.

\vspace{5mm}

Para la gestión de entornos virtuales se utilizó \emph{Conda} \cite{anaconda}, un sistema que permite crear espacios de trabajo aislados con versiones específicas de bibliotecas.
\vspace{5mm}

Sobre la base de \emph{Conda}, se implementó \emph{UV} \cite{uv2024github}, un gestor de paquetes y proyectos para \emph{Python}, extremadamente rápido y escrito en \emph{Rust} \cite{matsakis2014rust}. \emph{UV} fue utilizado para la instalación y gestión de paquetes dentro del entorno \emph{Conda}, aprovechando su capacidad para resolver dependencias de manera más eficiente y rápida que las herramientas tradicionales como \emph{pip} o el propio instalador de \emph{Conda}. Esta combinación permitió mantener un entorno consistente y reproducible mientras se optimizaba el tiempo de instalación y actualización de dependencias.

\vspace{5mm}

Para cada fase del proyecto se creó un paquete independiente con dependencias específicas según los requisitos de cada etapa. Las principales bibliotecas utilizadas en el proyecto incluyen:

\begin{itemize}
    \item \textbf{PyTorch con torchvision y torchaudio}: Framework principal para implementación de modelos de aprendizaje profundo \cite{NEURIPS2019_9015}. 
    \item \textbf{NumPy y SciPy}: Para operaciones numéricas y manipulación eficiente de matrices \cite{2020NumPy-Array, 2020SciPy-NMeth}.
    \item \textbf{Scikit-learn}: Para implementación de algoritmos de aprendizaje automático y métricas \cite{sklearn_api}.
    \item \textbf{Spectral}: Biblioteca especializada para procesamiento de imágenes hiperespectrales \cite{thomas_boggs_2022_7135091}.
    \item \textbf{OpenCV-Python}: Para operaciones de procesamiento de imágenes y visión por computador \cite{opencv_library}.
    \item \textbf{Transformers}: Para implementación y uso de modelos basados en arquitecturas de \emph{transformers} \cite{wolf-etal-2020-transformers}.
    \item \textbf{Timm}: Colección de modelos preentrenados para tareas de visión por computador \cite{rw2019timm}.
    \item \textbf{Supervision}: Para visualización y análisis de resultados de detección y segmentación \cite{Roboflow_Supervision}.
    \item \textbf{PyCocoTools}: Para manipulación de anotaciones en formato \emph{COCO} \cite{Welsh2018,Hoops2006,Medley2018}.
    \item \textbf{DEAP}: Para implementación de algoritmos genéticos y evolutivos \cite{DEAP_JMLR2012}.
\end{itemize}

\sloppy
Adicionalmente, se incorporaron bibliotecas auxiliares como \texttt{addict}, \texttt{colorlog}, \texttt{gdown}, \texttt{split-folders}, \texttt{submitit} y \texttt{termcolor} para tareas de gestión de configuración, logging, descarga de modelos pre-entrenados, organización de datos y paralelización de tareas.

\vspace{5mm}

La gestión precisa de versiones de estas dependencias resultó crítica para garantizar la compatibilidad entre componentes y estabilidad del entorno de desarrollo.

\subsection{Entorno de Desarrollo Integrado}

Para el desarrollo del código se empleó Visual Studio Code (VS Code) como entorno de desarrollo integrado.

\subsection{Infraestructura Computacional}

El desarrollo y ejecución del proyecto se realizó en un servidor de alto rendimiento proporcionado por la Universidad de Extremadura, con las siguientes especificaciones técnicas:

\begin{itemize}
    \item \textbf{Procesador}:  Intel(R) Xeon(R) Silver 4310 CPU @ 2.10GHz.
    \item \textbf{Memoria RAM}: 512 GB DDR4.
    \item \textbf{Acelerador Gráfico}: 4 × NVIDIA A100 con 40 GB de memoria VRAM cada una, de las cuales se utilizó una para la ejecución de los modelos de aprendizaje profundo.
    \item \textbf{Almacenamiento}: 4 TB en disco SSD NVMe.
\end{itemize}

\subsection{Adquisición de Imágenes Hiperespectrales}

Las imágenes fueron capturadas utilizando una cámara hiperespectral \emph{SPECIM}, específicamente el modelo \emph{FX10 VNIR}, cuyas características técnicas principales incluyen: resolución espacial de 1024 píxeles (800 ancho × 1024 alto), rango espectral de 400 nm a 1000 nm (visible y parte del infrarrojo cercano), 448 bandas espectrales, y un salto espectral de 1.339 nm.

\vspace{5mm}

El conjunto de datos comprende 320 higos cosechados de la plantación de la variedad calabacita ubicada en la \emph{Finca La Orden-Valdesequera} (38°51' N, 6°40' W, altitud 184 m) en Guadajira, España, donde \emph{CICYTEX} tiene su sede central. Las imágenes hiperespectrales se capturaron durante un período de 2 semanas, utilizando cada semana 160 higos cosechados en diferentes etapas de madurez.

\vspace{5mm}

Cada semana, 160 higos se dividieron en cuatro subconjuntos de aproximandamente 40 especímenes cada uno. El primer grupo correspondió a los controles sanos (Clase 0), mientras que los tres grupos siguientes fueron inoculados con concentraciones de $10^3$ UFC/mL (Clase 1), $10^5$ UFC/mL (Clase 2), y $10^7$ UFC/mL (Clase 3), respectivamente. El proceso de inoculación se realizó mediante inmersión del área durante aproximadamente 3 segundos, siguiendo el protocolo establecido por \emph{CICYTEX}.

\vspace{5mm}

Las imágenes hiperespectrales se capturaron después de la inoculación cada 24 horas durante cinco días consecutivos. Entre cada sesión de adquisición, las muestras se almacenaron en una cámara de incubación controlada a 25°C, con humedad relativa entre 80 y 90\% para promover el crecimiento fúngico. Cada clase consistió de 380 imágenes hiperespectrales, generando un total de 1520 imágenes hiperespectrales para el conjunto de datos completo.

\vspace{5mm}

Las imágenes hiperespectrales capturadas se almacenan en una estructura de directorios organizada que incluye múltiples archivos asociados a cada adquisición. A continuación, se describe el formato y contenido de los archivos principales:

\begin{itemize}
    \item \textbf{Archivos de datos hiperespectrales (\texttt{.hdr}, \texttt{.raw})}: 
    \begin{itemize}
        \item El archivo \texttt{.hdr} contiene metadatos descriptivos de la imagen, como dimensiones espaciales, número de bandas espectrales, rango espectral, y formato de datos.
        \item El archivo \texttt{.raw} almacena los datos espectrales en bruto, organizados en un formato binario que representa la intensidad de cada banda para cada píxel.
    \end{itemize}
    \item \textbf{Referencias de calibración (\texttt{DARKREF}, \texttt{WHITEREF})}: 
    \begin{itemize}
        \item Los archivos \texttt{DARKREF} y \texttt{WHITEREF} contienen las referencias oscura y blanca necesarias para la posterior corrección radiométrica de las imágenes hiperespectrales.
    \end{itemize}
    \item \textbf{Imagen \texttt{.png}}: 
    \begin{itemize}
        \item Este archivo representa una visualización en falso color \emph{RGB} generada a partir de tres bandas seleccionadas del cubo hiperespectral.
    \end{itemize}
    \item \textbf{Archivos de metadatos (\texttt{.xml})}: 
    \begin{itemize}
        \item Contienen información adicional sobre las condiciones de captura, como fecha, hora, y parámetros experimentales.
    \end{itemize}
\end{itemize}

Esta estructura permite un manejo eficiente de los datos, facilitando tanto la corrección radiométrica como la integración con el flujo de procesamiento automatizado.


\section{Localización y Segmentación de Figuras}

\subsection{Objetivo de la Fase}
La primera fase del proyecto consiste en la creación del conjunto de datos mediante la localización y segmentación automatizada de higos individuales sobre las imágenes creadas a través del falso color \emph{RGB}. El objetivo principal es generar anotaciones precisas en formato COCO \cite{lin2015microsoftcococommonobjects}, que incluyan cuadros delimitadores y máscaras de segmentación para cada higo detectado, y extraer los subcubos hiperespectrales radiométricamente corregidos correspondientes a cada fruto.

\vspace{5mm}

Esta fase es fundamental para el flujo de trabajo completo, ya que permite el aislamiento automatizado de las regiones de interés que servirán como imágenes de entrada para el entrenamiento y la inferencia de la \emph{CNN}. La precisión en esta etapa condiciona directamente la calidad de los datos que utilizará la red en las fases posteriores, por lo que resulta esencial garantizar su exactitud.

\subsection{Herramientas y Tecnologías Empleadas}

La implementación de esta fase se basa en la integración de modelos de visión por computador de última generación, complementados con librerías especializadas para el procesamiento de datos hiperespectrales y manipulación de anotaciones.

\subsubsection{Grounding DINO}

\emph{Grounding DINO} \cite{liu2023grounding,ren2024grounding} es un modelo de inteligencia artificial de última generación especializado en la detección de objetos en imágenes mediante el uso combinado de descripciones textuales e información visual, permitiendo un análisis multimodal avanzado. Gracias a su arquitectura basada en \emph{transformers} \cite{vaswani2023attentionneed} y técnicas de aprendizaje profundo, puede localizar y etiquetar objetos de interés, sin necesidad de entrenamiento específico para cada tipo de objeto, lo que lo hace altamente adaptable para tareas de detección abiertas o \emph{zero-shot} \cite{socher2013zeroshotlearningcrossmodaltransfer}.

\subsubsection{SAM2 (Segment Anything Model 2)}

\emph{SAM (Segment Anything Model)} \cite{kirillov2023segany, ravi2024sam2segmentimages} es un modelo de inteligencia artificial de última generación diseñado para segmentar cualquier objeto en imágenes o videos de manera automática y versátil. Fue desarrollado por \emph{Meta AI} y su objetivo principal es permitir la segmentación de objetos de imágenes y videos sin necesidad de entrenamiento específico para cada clase, usando tecnologías de visión por computador avanzadas y aprendizaje \emph{zero-shot}. Está entrenado en uno de los mayores conjuntos de datos existentes (SA-1B), con 11 millones de imágenes y 1.1 mil millones de máscaras de segmentación, lo que le da una capacidad sobresaliente para generalizar a nuevos contextos visuales.

\subsubsection{COCO (Common Objects in Context)}
El formato COCO (Common Objects in Context) \cite{lin2015microsoftcococommonobjects} es un estándar ampliamente adoptado para el almacenamiento y intercambio de anotaciones en tareas de visión por computador, especialmente en detección de objetos, segmentación de instancias y estimación de poses. Desarrollado por \emph{Microsoft Research},  \emph{COCO} define una estructura \emph{JSON} \cite{crockford2006application} que organiza metadatos de imágenes, anotaciones de objetos y categorías de manera eficiente y escalable.

\vspace{5mm}

Entre las componentes que definen la estructura del formato \emph{COCO}, se encuentran las \textbf{anotaciones (\emph{annotations})}, las cuales contienen las anotaciones específicas de cada objeto detectado, incluyendo identificadores únicos, referencias a imagen y categoría, coordenadas de bounding box, área, máscaras de segmentación en formato \emph{RLE (Run-Length Encoding)}, y banderas adicionales como \emph{iscrowd}, que indica si el objeto es parte de un grupo denso.

\vspace{5mm}

Para tareas de segmentación de instancias, las máscaras se codifican mediante \emph{RLE}, un algoritmo de compresión sin pérdidas que representa secuencias de píxeles consecutivos como pares (valor, longitud), reduciendo significativamente el espacio de almacenamiento requerido. Las coordenadas de bounding box se especifican en formato \texttt{[x, y, width, height]}, donde \texttt{(x, y)} representa la esquina superior izquierda del rectángulo delimitador.


\subsection{Flujo de Procesamiento}

La implementación del flujo de trabajo se diseñó siguiendo una arquitectura modular que separa conceptualmente la detección y anotación automatizada de la extracción de subcubos hiperespectrales. Esta separación se materializa en dos módulos principales: el primero responsable de la localización y segmentación de higos individuales sobre las imágenes RGB derivadas, y el segundo encargado de la extracción de los subcubos hiperespectrales correspondientes a cada detección validada.

\vspace{5mm}

El sistema adopta un patrón de procesamiento por lotes que opera sistemáticamente sobre la estructura jerárquica del conjunto de datos. Cada directorio de clase (C0, C1, C2, C3) contiene las imágenes hiperespectrales junto con sus archivos de metadatos y referencias de calibración.

\subsubsection{1. Detección}

El módulo de detección y segmentación constituye el núcleo del flujo automatizado. La implementación utiliza \emph{Grounding DINO} como modelo de detección, empleando la arquitectura con la red principal (\emph{backbone}) \emph{Swin Transformer Base} y el punto de control preentrenado correspondiente. Esta configuración permite al modelo procesar imágenes \emph{RGB} manteniendo su relación de aspecto original mientras utiliza la entrada de  texto \emph{fig} (higo en inglés) como descriptor semántico para guiar la detección.

\vspace{5mm}

La optimización de los parámetros de inferencia se estableció mediante experimentación empírica, fijando tanto el umbral de confianza de detección como el umbral de similaridad semántica texto-imagen en 0.25. Esta configuración proporciona un equilibrio óptimo entre sensibilidad de detección y precisión para el conjunto de datos específico, minimizando tanto los falsos positivos como los falsos negativos. El proceso de detección implementa una secuencia de validación que comienza con la carga de imágenes seguida de la conversión del espacio de color \emph{BGR} a \emph{RGB} y la extracción automática de metadatos temporales y experimentales del nombre del archivo.

\vspace{5mm}

Durante la inferencia, el modelo transforma las imágenes a tensores \emph{PyTorch} aplicando la normalización correspondiente a los parámetros del modelo preentrenado, ejecuta la detección con la entrada de texto especificada y aplica un filtrado geométrico crítico que limita las dimensiones máximas de los cuadros delimitadores a 250×150 píxeles. Esta restricción dimensional resulta fundamental para asegurar la detección de higos individuales y evitar regiones que abarquen múltiples especímenes, un problema recurrente en imágenes con alta densidad de objetos. El post-procesamiento convierte las coordenadas al formato requerido por \emph{SAM-2} y extrae las puntuaciones de confianza asociadas a cada detección.

\subsubsection{2. Segmentación}

La segmentación se realiza mediante \emph{SAM-2}, inicializado con la configuración \emph{Hiera Large} y el punto de control preentrenado correspondiente, utilizando el predictor específicamente diseñado para el procesamiento de imágenes estáticas. El modelo opera sin supervisión de puntos, empleando exclusivamente los cuadros delimitadores generados por \emph{Grounding DINO} como entrada primaria. La configuración para una única máscara por detección asegura la generación coherente, simplificando el procesamiento posterior y manteniendo la consistencia en las anotaciones. La figura \ref{fig:dino_sam} muestra un ejemplo representativo del proceso de detección y segmentación automatizada implementado.

\begin{figure}[ht]
\centering
\includegraphics[width=\textwidth]{images/dino_sam.jpg}
\caption{Ejemplo de detección y segmentación de higos utilizando \emph{Grounding DINO} y \emph{SAM-2}. La imagen muestra las cajas delimitadoras generadas por \emph{Grounding DINO} y las máscaras de segmentación producidas por \emph{SAM-2}.}
\label{fig:dino_sam}
\end{figure}

\subsubsection{3. Generación de Anotaciones}

La generación de anotaciones en formato \emph{COCO} se realizó mediante la construcción sistemática de estructuras de datos que incluyen metadatos de imagen, información de categorías y listas de anotaciones. Cada máscara binaria generada por \emph{SAM-2} se transforma al formato \emph{RLE} mediante las herramientas correspondientes, calculando automáticamente el área de cada instancia y asignando identificadores únicos secuenciales. El sistema exporta los resultados como archivos \emph{JSON} organizados por clase experimental, manteniendo la trazabilidad completa desde las imágenes originales hasta las anotaciones finales.

\subsubsection{4. Extracción de Subcubos Hiperespectrales}

El último componente del flujo se encarga de la extracción de sub-cubos hiperespectrales radiométricamente corregidos a partir de las detecciones validadas en la fase anterior. Este módulo implementa un procesamiento sofisticado que combina corrección radiométrica, mapeo geométrico y extracción volumétrica para generar datos hiperespectrales de alta calidad correspondientes a cada higo individual detectado.

\begin{figure}[ht]
\centering
\includegraphics[width=\textwidth]{images/higo_individual_220_113_78.png}
\caption{Ejemplo de la representación en falso color \emph{RGB} de un higo individual extraído del cubo hiperespectral. Las bandas espectrales utilizadas para la visualización son 220 (Rojo), 113 (Verde) y 78 (Azul).}
\label{fig:higo_individual}
\end{figure}


\vspace{5mm}

La corrección radiométrica constituye un paso crítico para garantizar la calidad de los datos hiperespectrales, eliminando efectos de iluminación y variaciones instrumentales que podrían comprometer el análisis posterior. El proceso utiliza las referencias oscura y blanca capturadas simultáneamente con cada imagen hiperespectral, aplicando la ecuación estándar de corrección:

\begin{equation}
R = \frac{RAW - DARK}{WHITE - DARK}
\end{equation}

donde $R$ representa la reflectancia corregida, $RAW$ los datos espectrales en bruto, $DARK$ la referencia oscura y $WHITE$ la referencia blanca. La implementación incluye el tratamiento robusto de casos especiales, como la prevención de divisiones por cero en regiones donde las referencias oscura y blanca presentan valores idénticos, situación que puede ocurrir en áreas de muy baja reflectancia.

\vspace{5mm}

La extracción geométrica de sub-cubos requiere una transformación precisa desde el espacio de coordenadas \emph{RGB}, donde se realizaron las detecciones, al espacio hiperespectral correspondiente. Esta conversión considera las posibles diferencias en resolución espacial entre las imágenes \emph{RGB} derivadas y los cubos hiperespectrales originales, implementando técnicas de mapeo que preservan la correspondencia espacial exacta. El sistema valida geométricamente cada región de interés para asegurar que los subcubos extraídos no excedan los límites físicos del cubo hiperespectral, evitando errores de indexación y garantizando la integridad de los datos espectrales.

\vspace{5mm}

El proceso de extracción volumétrica utiliza técnicas de división tridimensional optimizadas para mantener la estructura espectral completa de cada región de interés. Los subcubos resultantes preservan las 448 bandas espectrales originales junto con la resolución espacial correspondiente a cada detección, manteniendo la información espectral íntegra necesaria para el análisis posterior. La organización del almacenamiento sigue una estructura jerárquica específica, con subdirectorios organizados por clase experimental (\texttt{C0}, \texttt{C1}, \texttt{C2}, \texttt{C3}). Cada subcubo se almacena en formato \texttt{.npy} de \emph{NumPy} siguiendo una nomenclatura sistemática que incluye clase, timestamp y número de instancia, facilitando tanto el acceso eficiente como la trazabilidad completa mientras preserva la precisión numérica de punto flotante y optimiza los tiempos de carga durante el entrenamiento de modelos.

\subsection{Resultados}

La ejecución completa de la primera fase genera un conjunto estructurado de elementos que constituyen la base fundamental para las fases posteriores del proyecto. Las anotaciones \emph{COCO} resultantes comprenden archivos \emph{JSON} organizados por clase experimental, cada uno conteniendo metadatos completos de detección que incluyen coordenadas de cuadros delimitadores, máscaras de segmentación en formato \emph{RLE} y metadatos temporales extraídos automáticamente del sistema de nomenclatura implementado. Esta organización sistemática permite mantener la trazabilidad completa desde las imágenes originales hasta las detecciones finales, facilitando tanto la validación manual como el procesamiento automatizado en etapas subsecuentes.

\begin{figure}[H]
\centering
\small
\begin{verbatim}
{
    "info": {
        "description": "Fig detection and segmentation dataset",
        "version": "1.0",
        "year": 2024,
        "contributor": "Hyperspectral Analysis Pipeline",
        "date_created": "2024-01-15"
    },
    "licenses": [],
    "images": [
        {
            "id": 1,
            "width": 800,
            "height": 1024,
            "file_name": "C0_2023-07-17_10-15-30_001.png",
            "date_captured": "2023-07-17T10:15:30"
        }
    ],
    "annotations": [
        {
            "id": 1,
            "image_id": 1,
            "category_id": 1,
            "segmentation": {
                "size": [1024, 800],
                "counts": "nXh04M3M2N2N1O1O1N2N2N1O1O1N2N..."
            },
            "area": 12485,
            "bbox": [245.3, 412.7, 128.4, 97.2],
            "iscrowd": 0
        }
    ],
    "categories": [
        {
            "id": 1,
            "name": "fig",
            "supercategory": "fruit"
        }
    ]
}
\end{verbatim}
\caption{Ejemplo de anotación en formato COCO para la clase 0.}
\end{figure}

\subsection{Desafíos y Observaciones Técnicas}

Durante la implementación, se identificaron y resolvieron varios desafíos técnicos que proporcionaron valiosas lecciones para el desarrollo del proyecto. El primer y más significativo desafío encontrado fue determinar las versiones correctas de \emph{PyTorch} y sus dependencias relacionadas (\emph{torchvision} y \emph{torchaudio}) que fueran compatibles tanto con \emph{Grounding DINO} como con \emph{SAM-2}. Ambos modelos requieren versiones específicas del framework que no siempre coinciden, especialmente considerando las actualizaciones frecuentes en el ecosistema de aprendizaje profundo. La solución final involucró el análisis detallado de los requisitos de compatibilidad de cada modelo y la identificación de una versión común de \emph{PyTorch} 2.1.2 con soporte \emph{CUDA} 11.8 que proporcionara estabilidad y rendimiento óptimo para ambos componentes.

\vspace{5mm}

La optimización de la memoria de la \emph{GPU} constituyó otro reto importante, dado que el procesamiento conjunto de \emph{Grounding DINO} y \emph{SAM-2} requirió una implementación cuidadosa de contextos autocast para prevenir desbordamientos de memoria. La solución implementada aplica precisión mixta de forma selectiva \cite{micikevicius2018mixedprecisiontraining}: utiliza \texttt{torch.autocast} con \texttt{dtype=torch.bfloat16} únicamente para \emph{SAM-2}, mientras mantiene precisión completa para \emph{Grounding DINO}, equilibrando eficiencia computacional con calidad de inferencia. La elección del formato \texttt{bfloat16} se fundamenta en su diseño específico para aplicaciones de aprendizaje profundo, proporcionando un rango dinámico superior a \texttt{float16} tradicional \cite{8877411}. Adicionalmente, se configuró el uso de \emph{TF32} cuando está disponible en hardware compatible para optimizar las operaciones de multiplicación de matrices.

\section{Selección de Bandas}

\subsection{Objetivo de la Fase}
La segunda fase del proyecto se centra en la optimización de la selección de bandas espectrales mediante la implementación de un algoritmo genético que identifica automáticamente las tres bandas más informativas del cubo hiperespectral. El objetivo principal es reducir la dimensionalidad de los datos hiperespectrales de manera inteligente, preservando la información espectral crítica para la clasificación de contaminación por aflatoxinas mientras se optimiza la eficiencia computacional del entrenamiento de redes neuronales.

\vspace{5mm}

La reducción de 448 bandas espectrales a una combinación RGB de tres bandas estratégicamente seleccionadas permite aprovechar arquitecturas de CNN preentrenadas diseñadas para imágenes RGB convencionales, facilitando el uso de técnicas de aprendizaje por transferencia mediante ajuste fino selectivo \cite{NEURIPS2019_9015} con modelos robustos como \emph{ResNet-50} \cite{he2016deep} sin sacrificar la capacidad discriminativa del sistema.

\subsection{Herramientas y Tecnologías Empleadas}

La implementación de esta fase integra algoritmos evolutivos de optimización con técnicas de aprendizaje profundo, creando un sistema híbrido que combina la exploración global de los algoritmos genéticos con la capacidad de generalización de las redes neuronales convolucionales.

\subsubsection{DEAP (Distributed Evolutionary Algorithms in Python)}

\emph{DEAP} \cite{DEAP_JMLR2012} es un framework innovador de computación evolutiva diseñado específicamente para el prototipado rápido y la evaluación eficiente de ideas en el ámbito de la optimización bio-inspirada. A diferencia de otros frameworks tradicionales que imponen limitaciones mediante tipos predefinidos, \emph{DEAP} adopta una filosofía de diseño que prioriza la flexibilidad y la transparencia, permitiendo a los desarrolladores crear tipos de datos apropiados, personalizar inicializadores según sus necesidades específicas y seleccionar operadores de manera explícita y fundamentada.

\vspace{5mm}

En el contexto específico de este proyecto, \emph{DEAP} facilita la implementación de un algoritmo genético especializado para la selección de bandas hiperespectrales mediante la creación de tipos personalizados que representan combinaciones de índices espectrales, operadores de cruce y mutación que respetan las restricciones del dominio (rango [0, 447]), y estrategias de selección por torneo optimizadas para el problema de clasificación.

\subsubsection{ResNet-50}

ResNet-50 \cite{he2015deepresiduallearningimage} constituye la arquitectura base empleada para evaluar la aptitud de cada individuo en el algoritmo genético. Esta red neuronal convolucional profunda introduce el concepto de \emph{residual connections}, que permiten entrenar de forma estable modelos sustancialmente más profundos al mitigar problemas de desvanecimiento del gradiente. Con sus 50 capas, \emph{ResNet-50} logra un equilibrio entre profundidad y eficiencia computacional, ofreciendo un rendimiento robusto en tareas de clasificación de imágenes. Al estar preentrenada en el extenso conjunto de datos ImageNet \cite{deng2009imagenet}, proporciona un punto de partida sólido para el aprendizaje por transferencia, lo que permite aprovechar representaciones visuales generales y adaptarlas a la tarea específica de clasificación de imágenes \emph{RGB} derivadas de las combinaciones de bandas espectrales seleccionadas por el algoritmo evolutivo.

\subsection{Flujo de Procesamiento}

El algoritmo genético implementado utiliza una arquitectura evolutiva estándar, adaptada específicamente para abordar el problema de selección de bandas hiperespectrales. Cada iteración del proceso evolutivo integra una evaluación basada en aprendizaje profundo, optimizando así la selección de bandas de manera eficiente.

\subsubsection{1. Representación y Inicialización}

Cada individuo en la población se representa mediante un vector de tres enteros en el rango [0, 447], donde cada posición del vector corresponde a una banda espectral específica: el primer entero representa la banda asignada al canal \emph{R} (rojo), el segundo entero al canal \emph{G} (verde), y el tercer entero al canal \emph{B} (azul). Estas bandas seleccionadas se combinan para formar la imagen \emph{RGB}. La población inicial se genera aleatoriamente con un tamaño de 20 individuos, valor determinado mediante experimentación empírica para equilibrar diversidad poblacional con eficiencia computacional. La inicialización uniforme garantiza la exploración inicial del espacio completo de 448 bandas disponibles, evitando sesgos hacia regiones específicas del espectro electromagnético.

\subsubsection{2. Función de Evaluación}

La evaluación de la aptitud constituye el componente más computacionalmente intensivo del algoritmo, requiriendo el entrenamiento mediante ajuste fino parcial de una red \emph{ResNet-50} para cada individuo evaluado. El proceso comienza con la construcción de imágenes \emph{RGB} utilizando las tres bandas especificadas por el individuo, seguido de la creación de cargadores de datos de \emph{PyTorch} para los conjuntos de entrenamiento y prueba.

\vspace{5mm}

El modelo \emph{ResNet-50} preentrenado en \emph{ImageNet} se adapta específicamente para el problema de clasificación de cuatro clases mediante una estrategia de aprendizaje por transferencia basada en ajuste fino selectivo. La arquitectura implementa las siguientes modificaciones: (1) congelación de todas las capas desde la entrada hasta \texttt{layer3} (inclusive), manteniendo los pesos preentrenados para la extracción de características de bajo y medio nivel; (2) liberación de los parámetros de \texttt{layer4}, la capa convolucional más profunda, permitiendo la adaptación de características de alto nivel específicas para la clasificación de higos con diferentes niveles de contaminación; y (3) reemplazo completo de la capa de clasificación final (\texttt{fc}) por una nueva capa lineal con 4 unidades de salida correspondientes a las clases \emph{C0}, \emph{C1}, \emph{C2} y \emph{C3}, inicializada aleatoriamente.

\vspace{5mm}

Esta configuración resulta en un modelo con aproximadamente 2.3 millones de parámetros entrenables de los 25.6 millones totales, concentrando el aprendizaje en las representaciones más específicas del dominio mientras preserva las características generales aprendidas en \emph{ImageNet}. El entrenamiento se ejecuta durante 50 épocas utilizando el optimizador \emph{Adam} con una tasa de aprendizaje de 0.001, aplicando técnicas de aumento de datos que incluyen volteos horizontales y verticales aleatorios, rotaciones de hasta 15 grados, y normalización estándar de \emph{ImageNet}. La aptitud final del individuo se define como la precisión de prueba alcanzada en la última época, proporcionando una medida directa de la capacidad clasificatoria de la combinación de bandas espectrales seleccionada.

\vspace{5mm}

Para optimizar la eficiencia computacional, se implementó un sistema de caché que almacena los resultados de evaluaciones previas, evitando el reentrenamiento de combinaciones de bandas ya evaluadas en generaciones anteriores.

\subsubsection{3. Operadores Genéticos}

El algoritmo implementa operadores genéticos especializados que respetan las restricciones del dominio espectral y aprovechan las características intrínsecas de la información hiperespectral. La selección de estos operadores se fundamenta en el principio de que la información espectral relevante para la clasificación tiende a concentrarse en bandas espectralmente adyacentes debido a la correlación espacial natural entre longitudes de onda vecinas en el espectro electromagnético.

\vspace{5mm}

\textbf{Operador de Cruce:} El cruce utiliza una variante modificada del operador de mezcla (\emph{blend crossover}) que combina linealmente los valores de los padres. La fórmula utilizada es:

\begin{equation}
hijo_i = (1-\gamma) \cdot padre1_i + \gamma \cdot padre2_i
\end{equation}

donde $\gamma$ se genera aleatoriamente en cada posición. Este operador resulta particularmente apropiado para la selección de bandas hiperespectrales porque produce descendientes cuyos índices de banda se mantienen en regiones espectrales intermedias entre los padres, preservando la localidad espectral y evitando saltos abruptos hacia bandas distantes que podrían no contener información correlacionada. Tras la aplicación del operador, se aplican restricciones de dominio para garantizar que los descendientes mantengan índices válidos en el rango [0, 447].

\vspace{5mm}

\textbf{Operador de Mutación:} La mutación emplea un operador gaussiano con desviación estándar de 1.0 y probabilidad individual de 0.1, seguido de restricción al dominio válido. La elección de la mutación gaussiana con desviación estándar reducida ($\sigma=1.0$) está específicamente diseñada para introducir variaciones locales que exploren bandas espectralmente cercanas a las actuales, aprovechando el hecho de que bandas adyacentes en el espectro electromagnético típicamente contienen información complementaria y correlacionada. Esta estrategia de mutación conservativa evita perturbaciones drásticas que podrían llevar la búsqueda hacia regiones espectrales completamente diferentes y potencialmente menos informativas, manteniendo la continuidad espectral mientras permite la exploración gradual del espacio de soluciones.

\vspace{5mm}

\textbf{Selección y Elitismo:} La selección de padres utiliza torneos de tamaño 3, proporcionando una presión selectiva moderada que equilibra la explotación de buenas soluciones con la exploración de nuevas regiones del espacio de búsqueda. Se implementa elitismo con tamaño 1, garantizando la preservación de la mejor solución encontrada a través de las generaciones y asegurando que el algoritmo no pierda combinaciones de bandas de alta aptitud durante el proceso evolutivo.

\subsubsection{4. Configuración Experimental}

El algoritmo se configura para ejecutar durante 50 generaciones con una población de 20 individuos, utilizando probabilidades de cruzamiento y mutación de 0.8 y 0.15 respectivamente. Estos parámetros se determinaron mediante experimentación previa para optimizar el balance entre exploración y explotación en el contexto específico del problema de selección de bandas.

\vspace{5mm}

Para garantizar la reproducibilidad y robustez estadística, se implementó un sistema de experimentos múltiples que permite la ejecución de varias corridas independientes del algoritmo, cada una con semillas aleatorias diferentes. Los resultados de cada experimento se almacenan automáticamente en archivos CSV que incluyen estadísticas generacionales completas y métricas detalladas del mejor individuo.

\subsection{Resultados}

Se ejecutaron diez experimentos independientes. Los resultados demuestran variaciones significativas en el rendimiento clasificatorio según las bandas seleccionadas.

\vspace{5mm}

La Tabla \ref{tab:experiment_metrics} presenta el rendimiento de cada experimento. Las precisiones de prueba oscilaron entre 0.78 y 0.84, con una variabilidad de 6 puntos porcentuales que subraya la importancia de la selección inteligente de bandas.

\begin{table}[ht]
\centering
\caption{Resumen de métricas de los experimentos de selección de bandas.}
\label{tab:experiment_metrics}
\resizebox{\textwidth}{!}{
\csvautotabular{tables/experiment_metrics_summary.csv}
}
\end{table}

El Experimento 10 alcanzó el rendimiento superior con precisión de 0.84 utilizando las bandas [220, 113, 78], seguido por los Experimentos 2, 3, 4, 6 y 7 con precisión de 0.82. Los Experimentos 8 y 9 mostraron el rendimiento más bajo (0.78 y 0.79), indicando que sus combinaciones espectrales fueron menos efectivas para la discriminación de contaminación. La Figura \ref{fig:exp_10_cnn_results_220_113_78} muestra las curvas de entrenamiento del experimento de mejor rendimiento, mientras que la Tabla \ref{tab:exp_10_cnn_results_220_113_78} presenta sus métricas detalladas.

\begin{figure}[ht]
\centering
\includegraphics[width=\textwidth]{images/exp_10_cnn_results_220_113_78.png}
\caption{Curvas de precisión y pérdida del experimento 10 con las bandas seleccionadas 220, 113 y 78.}
\label{fig:exp_10_cnn_results_220_113_78}
\end{figure}

\begin{table}[ht]
\centering
\caption{Resultados del experimento 10 con las bandas seleccionadas 220, 113 y 78.}
\label{tab:exp_10_cnn_results_220_113_78}
\resizebox{\textwidth}{!}{
\csvautotabular{tables/exp_10_cnn_results_220_113_78.csv}
}
\end{table}

Las métricas \emph{F1} macro mantuvieron consistencia con las tendencias de precisión (0.75-0.82), confirmando capacidades de discriminación equilibradas entre las cuatro categorías de contaminación sin sesgos hacia clases específicas, como se detalla en la Tabla \ref{tab:exp_10_classification_report_220_113_78}.

\begin{table}[ht]
\centering
\caption{Reporte de clasificación del experimento 10 con las bandas seleccionadas 220, 113 y 78.}
\label{tab:exp_10_classification_report_220_113_78}
\resizebox{\textwidth}{!}{
\csvautotabular{tables/exp_10_classification_report_220_113_78.csv}
}
\end{table}

El análisis de la evolución del fitness revela un patrón de convergencia consistente en todos los experimentos, con tres fases características. La \textbf{fase inicial} (generaciones 0-10) muestra mejora rápida del fitness promedio desde 0.69 hasta 0.79 con alta variabilidad poblacional. La \textbf{fase de convergencia} (generaciones 10-25) estabiliza el fitness promedio en 0.82-0.84 con reducción gradual de variabilidad. La \textbf{fase final} (generaciones 25-50) exhibe convergencia completa hacia el valor óptimo de 0.84, confirmando la estabilidad de las soluciones identificadas y la robustez del algoritmo evolutivo, como se observa en la Figura \ref{fig:exp_10_fitness_evolution}.

\begin{figure}[ht]
\centering
\includegraphics[width=\textwidth]{images/exp_10_fitness_evolution.png}
\caption{Curvas de evolución de la aptitud del experimento 10 con las bandas seleccionadas 220, 113 y 78.}
\label{fig:exp_10_fitness_evolution}
\end{figure}

El Experimento 10 representa la configuración de máximo rendimiento, utilizando las bandas 220, 113 y 78. El reporte de clasificación detallado (Tabla \ref{tab:exp_10_classification_report_220_113_78}) muestra:

\begin{itemize} \item \textbf{C0 (Saludables):} Precisión = 0.797, Recall = 0.839, F1 = 0.817 \item \textbf{C1 (Baja contaminación):} Precisión = 0.863, Recall = 0.786, F1 = 0.822 \item \textbf{C2 (Media contaminación):} Precisión = 0.825, Recall = 0.825, F1 = 0.825 \item \textbf{C3 (Alta contaminación):} Precisión = 0.797, Recall = 0.825, F1 = 0.810 \end{itemize}

La precisión global alcanzó 81.9\% con promedios macro de 0.820, 0.819 y 0.819 para precisión, recall y F1-score respectivamente. La matriz de confusión (Tabla \ref{tab:exp_10_confusion_matrix_220_113_78}) reveló tasas de detección verdadero positivo robustas: C0 (83.9\%), C1 (78.6\%), C2 (82.5\%) y C3 (82.5\%), con errores de clasificación distribuidos uniformemente sin sesgos sistemáticos.

\begin{table}[ht]
\centering
\caption{Matriz de confusión del experimento 10 con las bandas seleccionadas 220, 113 y 78.}
\label{tab:exp_10_confusion_matrix_220_113_78}
\resizebox{\textwidth}{!}{
\csvautotabular{tables/exp_10_confusion_matrix_220_113_78.csv}
}
\end{table}

Los resultados demuestran la efectividad de la metodología híbrida que combina algoritmos genéticos con redes neuronales convolucionales para la selección automática de bandas espectrales. La convergencia consistente observada entre las generaciones 10-25 confirma la eficiencia del algoritmo evolutivo para identificar soluciones óptimas evitando la exploración exhaustiva de $\binom{448}{3} \approx 1.5 \times 10^7$ combinaciones posibles.

\subsection{Desafíos y Observaciones Técnicas}

Durante la implementación de esta fase del proyecto, se identificaron varios desafíos técnicos significativos que impactaron tanto en los tiempos de desarrollo como en la metodología experimental. Estos desafíos proporcionan conocimientos valiosos para futuras implementaciones y optimizaciones del sistema.

\vspace{5mm}

Uno de los principales obstáculos técnicos encontrados se relacionó con la configuración de los cargadores de datos de PyTorch (\texttt{DataLoader}). Durante las pruebas iniciales, se observó que configurar el parámetro \texttt{num\_workers} con valores superiores a 0 resultaba en tiempos de ejecución significativamente prolongados, contrario a las expectativas teóricas de paralelización. Este comportamiento aparentemente contradictorio se debe a varias consideraciones específicas del entorno de ejecución. En sistemas con limitaciones de memoria RAM o configuraciones específicas de hardware, la creación de múltiples procesos trabajadores puede introducir una sobrecarga significativa debido al costo de inicialización de procesos, transferencia de datos entre procesos padre e hijos, y sincronización. Esta sobrecarga puede superar los beneficios de la paralelización, especialmente cuando el conjunto de datos no es lo suficientemente grande o complejo como para justificar la distribución del trabajo.

\vspace{5mm}

Además, el uso de múltiples procesos trabajadores puede generar contención por recursos del sistema como acceso a disco, memoria compartida, y ancho de banda de memoria, lo que paradójicamente puede ralentizar el proceso de carga de datos. En el contexto de este proyecto, donde cada muestra requiere la carga y procesamiento de datos hiperespectrales de alta dimensionalidad (448 bandas espectrales), la contención de recursos se vuelve particularmente crítica. Este problema se ve agravado en entornos de servidores compartidos como este, donde múltiples usuarios ejecutan trabajos simultáneamente, ya que aunque la GPU pueda estar dedicada exclusivamente al proceso de entrenamiento, otros recursos críticos como CPU, memoria RAM, ancho de banda de memoria y sistema de I/O de disco pueden estar bajo demanda concurrente, incrementando significativamente la latencia de operaciones de carga de datos paralelas. La solución adoptada fue mantener \texttt{num\_workers=0}, configuración que fuerza la carga secuencial de datos en el proceso principal. Aunque esta configuración elimina el paralelismo en la carga de datos, resultó en tiempos de entrenamiento significativamente más eficientes para las características específicas del conjunto de datos y la infraestructura utilizada.

\vspace{5mm}

La definición de la arquitectura óptima del modelo constituyó un proceso iterativo y computacionalmente intensivo que requirió varias semanas de experimentación sistemática. Este proceso involucró la exploración de múltiples configuraciones arquitectónicas, cada una requiriendo ciclos completos de entrenamiento para su evaluación. Se evaluaron diferentes estrategias de congelación de capas en la red ResNet-50 preentrenada, incluyendo: (1) congelación completa de la estructura base con entrenamiento únicamente de la capa clasificadora final; (2) liberación progresiva de capas desde la capa de clasificación hacia capas convolucionales más profundas; y (3) configuraciones híbridas que combinan congelación selectiva con tasas de aprendizaje diferenciadas. La configuración final, que congela capas hasta \texttt{layer3} inclusive y libera \texttt{layer4} y la capa \texttt{fc}, emergió como el equilibrio óptimo entre capacidad de adaptación al dominio específico y preservación de características preentrenadas útiles.

\vspace{5mm}

Se exploraron arquitecturas alternativas incluyendo variantes de EfficientNet, DenseNet, y configuraciones personalizadas de ResNet con diferentes profundidades. Cada arquitectura requirió ajustes específicos de hiperparámetros, estrategias de regularización, y técnicas de aumento de datos, resultando en un proceso experimental extensivo que, aunque costoso temporalmente, fue fundamental para identificar la configuración óptima.

\section{Análisis Hiperespectral mediante Transformadas Wavelet}

\subsection{Objetivo de la Fase}

La tercera fase del proyecto implementa una metodología alternativa para el análisis de imágenes hiperespectrales mediante la aplicación de transformadas \emph{wavelet} en regiones espacialmente localizadas del higo. Esta aproximación sigue la línea de investigación establecida por Cruz-Carrasco et al. \cite{agriengineering6040225}, quienes demostraron la efectividad del análisis hiperespectral combinado con transformadas \emph{wavelet} para la detección de \emph{Aspergillus flavus} en higos. Sin embargo, mientras que el estudio de referencia realizó el análisis a nivel de píxel individual, esta fase extiende la metodología considerando regiones espaciales más amplias que capturan información contextual y variabilidad espacial local.

\vspace{5mm}

El objetivo principal es explorar la capacidad de las transformadas \emph{wavelet} continuas (CWT) y discretas (DWT) para extraer características espectrales discriminativas a partir de regiones de 32×32 píxeles subdivididas en parches de 4×4 píxeles, permitiendo la clasificación efectiva de diferentes niveles de contaminación por aflatoxinas. Esta aproximación regional, en contraste con el análisis puntual por píxeles, permite capturar patrones espaciales de heterogeneidad espectral que pueden indicar diferentes etapas del desarrollo fúngico y la distribución espacial de la contaminación dentro del tejido del higo.

\vspace{5mm}

Esta aproximación se fundamenta en la hipótesis de que las transformadas \emph{wavelet} pueden capturar información espectral-temporal crítica que métodos tradicionales de procesamiento espectral podrían no detectar. A diferencia de las fases anteriores que operan sobre imágenes \emph{RGB} derivadas o selecciones optimizadas de bandas espectrales, esta metodología procesa directamente la información espectral completa mediante análisis tiempo-frecuencia, preservando tanto las características espectrales como sus variaciones temporales a lo largo del espectro electromagnético.

\subsection{Herramientas y Tecnologías Empleadas}

La implementación de esta fase integra técnicas avanzadas de procesamiento de señales con arquitecturas de aprendizaje profundo especializadas, combinando análisis \emph{wavelet} con redes neuronales convolucionales para la clasificación automatizada de muestras.

\subsubsection{PyWavelets}

\emph{PyWavelets} \cite{lee2019pywavelets} es una biblioteca de Python de código abierto que implementa de manera eficiente y numéricamente estable una amplia gama de transformadas \emph{wavelet}, tanto en su versión discreta como continua. Ofrece soporte para transformadas multiescala, transformadas inversas, y herramientas de análisis como filtrado, compresión y descomposición de señales. La biblioteca incluye múltiples familias de \emph{wavelets} (\emph{Daubechies}, \emph{Morlet}, \emph{Haar}, \emph{Biorthogonal}, entre otras), lo que permite seleccionar la función base más adecuada según la aplicación. En el marco de este proyecto, \emph{PyWavelets} se emplea para transformar firmas espectrales unidimensionales en representaciones tiempo-frecuencia bidimensionales (escalogramas), que capturan de forma localizada tanto las características frecuenciales como su posición en el dominio espectral.

\subsubsection{DenseNet-121}

La arquitectura \emph{DenseNet-121} \cite{huang2018denselyconnectedconvolutionalnetworks} se caracteriza por el uso de conexiones densas entre capas, en las cuales cada capa recibe como entrada no solo la salida de la capa inmediatamente anterior, sino también los mapas de características de todas las capas previas, lo que fomenta una reutilización eficiente de la información aprendida. Esta estrategia de conectividad permite mitigar el problema del desvanecimiento del gradiente, favoreciendo un entrenamiento más estable y profundo, al tiempo que reduce la redundancia de parámetros en comparación con otras arquitecturas tradicionales como \emph{ResNet}, gracias a la mayor eficiencia en el flujo de información. \emph{DenseNet-121}, con sus 121 capas organizadas en bloques densos e intercaladas con capas de transición que controlan la dimensionalidad y la complejidad computacional, logra un equilibrio entre profundidad, capacidad de representación y eficiencia, lo que la convierte en una de las arquitecturas más utilizadas en tareas de clasificación y extracción de características en el dominio del aprendizaje profundo.

\subsection{Flujo de Procesamiento}

El flujo de procesamiento implementa una metodología sistemática que combina extracción espacial de regiones de interés, análisis espectral mediante transformadas \emph{wavelet}, y clasificación mediante aprendizaje profundo.

\subsubsection{1. Extracción de Parches Espectrales}

El proceso comienza con la identificación automática de regiones óptimas de 32×32 píxeles en cada imagen hiperespectral. El algoritmo implementa una estrategia de búsqueda espacial que localiza la región centrada verticalmente y posicionada horizontalmente para minimizar la presencia de píxeles de fondo. Esta selección automatizada garantiza que los parches extraídos contengan exclusivamente información espectral del tejido del higo, eliminando interferencias del fondo de la imagen que podrían introducir ruido en el análisis posterior.

\vspace{5mm}

Una vez identificada la región óptima de 32×32 píxeles, se procede a la subdivisión sistemática en una grilla regular de 64 sub-parches de 4×4 píxeles cada uno. Para cada sub-parche de 4×4 píxeles, se extrae la firma espectral promedio calculando la media aritmética de los valores espectrales a través de las dimensiones espaciales. Este procedimiento de promediado espacial genera 64 firmas espectrales representativas, cada una de 448 bandas espectrales, que capturan la información espectral característica de regiones espacialmente localizadas dentro del higo individual.

\begin{figure}[ht]
\centering
\includegraphics[width=\textwidth]{images/patch_visualization_Fx10_20230717_Riego_C0_2023-07-17_09-02-21_ann0.png}
\caption{Visualización de la región de 32×32 píxeles extraída automáticamente del higo, subdividida en 64 parches de 4×4 píxeles.}
\label{fig:patch_extraction_example}
\end{figure}


Esta metodología de promediado espacial permite reducir el ruido espectral inherente a nivel de píxel individual mientras preserva las características espectrales distintivas de cada región. Al calcular la media espectral de los 16 píxeles (4×4) que componen cada sub-parche, se obtiene una representación espectral más robusta y estable que facilita la identificación de patrones espectrales asociados con diferentes grados de contaminación fúngica.

\begin{figure}[ht]
\centering
\includegraphics[width=\textwidth]{images/hyperspectral_signature.png}
\caption{Ejemplo de firma espectral promedio extraída de un parche de 4×4 píxeles. La gráfica muestra la reflectancia en función de las 448 bandas espectrales.}
\label{fig:hyperspectral_signature_example}
\end{figure}

\subsubsection{2. Transformada Wavelet Continua}

La transformación espectral utiliza la Transformada Wavelet Continua (\emph{CWT}) con la \emph{wavelet} de \emph{Morlet} como función base. Esta elección se fundamenta en las propiedades óptimas de la \emph{wavelet} de \emph{Morlet} para análisis espectral: proporciona una resolución tiempo-frecuencia balanceada, mantiene fase constante, y exhibe características de localización espectral apropiadas para señales hiperespectrales. La \emph{CWT} se aplica a cada firma espectral de 448 bandas utilizando 64 escalas logarítmicamente distribuidas, generando escalogramas de 64×448 píxeles que representan la distribución tiempo-frecuencia de la información espectral.

\begin{figure}[ht]
\centering
\includegraphics[width=\textwidth]{images/CWT_morl_Fx10_20230717_Riego_C0_2023-07-17_09-02-21_ann0_patch_00.png}
\caption{Ejemplo de escalograma generado por la Transformada Wavelet Continua (\emph{CWT}) utilizando la \emph{wavelet} de \emph{Morlet}.}
\label{fig:cwt_morl_example}
\end{figure}

\vspace{5mm}

Los escalogramas resultantes capturan tanto características espectrales globales como variaciones localizadas en frecuencia, proporcionando una representación rica en información que preserva patrones espectrales discriminativos para la detección de contaminación por aflatoxinas. La selección de 64 escalas permite una cobertura completa del rango espectral mientras mantiene una resolución suficiente para detectar características espectrales finas asociadas con cambios bioquímicos inducidos por el desarrollo fúngico.

\subsubsection{3. Preparación de Datos para CNN}

Los 64 escalogramas generados por cada parche de 32×32 píxeles se organizan como imágenes individuales para el entrenamiento de la red neuronal convolucional. Esta estrategia de aumentado de datos resulta en una expansión significativa del conjunto de datos: cada imagen hiperespectral original genera 64 instancias de entrenamiento, multiplicando efectivamente el tamaño del conjunto de datos disponible para el aprendizaje supervisado.

\vspace{5mm}

Los escalogramas se normalizan utilizando estadísticas estándar de \emph{ImageNet} para aprovechar las ventajas del aprendizaje por transferencia. La organización del conjunto de datos sigue una estructura jerárquica estándar con directorios separados para entrenamiento y prueba, manteniendo la distribución de clases balanceada (\emph{C0}, \emph{C1}, \emph{C2}, \emph{C3}) y preservando la trazabilidad desde las imágenes hiperespectrales originales hasta los escalogramas individuales.

\subsubsection{4. Arquitectura y Entrenamiento de CNN}

La red \emph{DenseNet-121} se adapta específicamente para la clasificación de cuatro clases mediante la modificación de la capa clasificadora final. La arquitectura implementa una cabeza clasificadora personalizada que incluye capas lineales con normalización por lotes, funciones de activación \emph{ReLU}, y regularización mediante \emph{dropout} para prevenir sobreajuste. La configuración final comprende: una capa lineal de 1024 a 128 unidades con \emph{ReLU} y normalización por lotes, seguida de \emph{dropout} (0.4), una segunda capa lineal de 128 a 64 unidades con \emph{ReLU} y \emph{dropout} (0.3), y finalmente una capa de salida de 64 a 4 unidades correspondientes a las clases experimentales.

\vspace{5mm}

El entrenamiento se ejecuta durante 50 épocas utilizando el optimizador \emph{Adam} con tasa de aprendizaje inicial de 0.001 y un planificador \emph{ReduceLROnPlateau} que reduce la tasa de aprendizaje cuando la pérdida de validación se estanca. El modelo utiliza función de pérdida de entropía cruzada y técnicas de aumento de datos que incluyen transformaciones estocásticas apropiadas para imágenes de escalogramas, preservando las características tiempo-frecuencia mientras introducen variabilidad para mejorar la generalización.

\begin{table}[ht]
\centering
\caption{Configuración de la cabeza clasificadora personalizada para \emph{DenseNet-121}.}
\label{tab:custom_layers}
\resizebox{\textwidth}{!}{
\csvautotabular{tables/custom_layers.csv}
}
\end{table}

\subsection{Resultados}

La metodología basada en transformadas wavelet demostró una efectividad notable para la clasificación de contaminación por aflatoxinas en muestras de higo. El modelo \emph{DenseNet-121} entrenado alcanzó una precisión de clasificación de \textbf{87.43\%} en el conjunto de prueba, con métricas consistentes a través de las cuatro clases experimentales.

\begin{figure}[ht]
\centering
\includegraphics[width=\textwidth]{images/wavelet_training_results.png}
\caption{Curvas de entrenamiento del modelo \emph{DenseNet-121} para clasificación de escalogramas wavelet. Se muestran la evolución de la pérdida y precisión durante 50 épocas de entrenamiento.}
\label{fig:wavelet_training_results}
\end{figure}

\vspace{5mm}

El análisis detallado por clases revela un rendimiento balanceado: \emph{C0} (control sano) alcanzó precisión de 88.92\% y recall de 88.59\%, \emph{C1} (baja contaminación) obtuvo precisión y recall de 88.26\%, \emph{C2} (contaminación media) logró precisión de 85.24\% y recall de 84.49\%, mientras que \emph{C3} (alta contaminación) registró precisión de 87.35\% y recall de 88.43\%. Los puntajes F1 correspondientes fueron 88.76\%, 88.26\%, 84.86\% y 87.89\% respectivamente, indicando un equilibrio efectivo entre precisión y recall en todas las categorías.

\begin{table}[ht]
\centering
\caption{Reporte de clasificación del modelo basado en transformadas wavelet.}
\label{tab:wavelet_report}
\resizebox{\textwidth}{!}{
\csvautotabular{tables/wavelet_classification_report.csv}
}
\end{table}

\vspace{5mm}

La matriz de confusión revela patrones de clasificación interpretables: las confusiones más frecuentes ocurren entre clases adyacentes (\emph{C1-C2} y \emph{C2-C3}), reflejando la progresión gradual de la contaminación por aflatoxinas. Específicamente, se observan 190 confusiones entre \emph{C1} y \emph{C2}, y 174 confusiones entre \emph{C2} y \emph{C3}. La clase control (\emph{C0}) muestra la menor tasa de confusión con clases contaminadas, con solo 403 clasificaciones erróneas de un total de 3533 muestras, indicando que la metodología wavelet es efectiva para distinguir muestras sanas de contaminadas.

\begin{table}[ht]
\centering
\caption{Matriz de confusión del modelo basado en transformadas wavelet.}
\label{tab:wavelet_confusion_matrix}
\resizebox{\textwidth}{!}{
\csvautotabular{tables/wavelet_confusion_matrix.csv}
}
\end{table}

\vspace{5mm}

El entrenamiento del modelo se completó en 50 épocas, con un tiempo total de entrenamiento de aproximadamente 3.1 horas en una GPU NVIDIA A100. La convergencia del modelo se observa claramente en las curvas de entrenamiento, donde tanto la pérdida como la precisión se estabilizan después de las primeras 20 épocas, indicando un aprendizaje efectivo sin evidencia significativa de sobreajuste.

\subsection{Desafíos y Observaciones Técnicas}

La implementación de la metodología wavelet presentó varios desafíos técnicos que proporcionaron perspectivas valiosas sobre el procesamiento de imágenes hiperespectrales. El primer desafío significativo fue la selección de parámetros óptimos para la CWT, particularmente el número de escalas y su distribución. La experimentación inicial con diferentes configuraciones reveló que 64 escalas logarítmicamente distribuidas proporcionan el mejor equilibrio entre resolución espectral y eficiencia computacional, capturando características espectrales finas sin introducir redundancia excesiva.

\vspace{5mm}

La gestión de memoria constituyó otro aspecto crítico, dado que la generación de escalogramas para el conjunto completo de datos resulta en un volumen significativo de datos procesados. La implementación de procesamiento por lotes y técnicas de liberación explícita de memoria (\texttt{gc.collect()}) después de cada lote de procesamiento permitió manejar eficientemente el flujo de datos sin saturar la memoria disponible. Adicionalmente, se observó que la normalización apropiada de los escalogramas resulta crítica para la convergencia estable del entrenamiento, requiriendo ajustes específicos para accommodar las características únicas de las representaciones tiempo-frecuencia generadas por CWT.

\vspace{5mm}

Un hallazgo técnico importante fue que la metodología wavelet captura información espectral complementaria a las aproximaciones basadas en selección de bandas RGB. Los escalogramas revelan patrones espectrales localizados que no son evidentes en análisis espectrales tradicionales, sugiriendo que las transformadas wavelet proporcionan una perspectiva única para la caracterización de cambios bioquímicos asociados con el desarrollo fúngico y la producción de aflatoxinas en tejidos vegetales.


\chapter{Resultados}

En este capítulo se presentan los resultados obtenidos en cada una de las fases del desarrollo del sistema de detección de aflatoxinas en higos frescos mediante a% Temporarily co% Temporarily removed for debuggingto compilation issue
% \begin{figure}[!ht]
% \centering
% \begin{subfigure}[b]{0.45\textwidth}
%     \includegraphics[width=\textwidth]{images/mutacion1.png}
%     \caption{Mutación por intercambio}
%     \label{fig:mutacion_intercambio}
% \end{subfigure}
% \hfill
% \begin{subfigure}[b]{0.45\textwidth}
%     \includegraphics[width=\textwidth]{images/mutacion2.png}
%     \caption{Mutación por inversión}
%     \label{fig:mutacion_inversion}
% \end{subfigure}
% \caption{Operadores de mutación implementados para introducir diversidad genética y evitar la convergencia prematura hacia óptimos locales en la selección de bandas espectrales.}
% \label{fig:operadores_mutacion}
% \end{figure}
% nes hiperespectrales. Los resultados se organizan de acuerdo con las fases metodológicas descritas en el capítulo anterior, proporcionando una evaluación comprehensiva del rendimiento del sistema desarrollado.

\section{Fase 0: Resultados de Detección y Segmentación}

\subsection{Rendimiento del Sistema de Detección}

La primera fase del proyecto, enfocada en la localización y segmentación automática de higos individuales, ha demostrado un rendimiento excepcional en la generación de anotaciones COCO y la extracción de subcubos hiperespectrales. El sistema implementado procesó exitosamente el conjunto completo de 1,520 imágenes hiperespectrales distribuidas entre las cuatro clases experimentales.

\subsubsection{Métricas de Detección con Grounding DINO}

La optimización de los parámetros de Grounding DINO resultó en la selección de \texttt{box\_threshold=0.25} y \texttt{text\_threshold=0.25}, valores que proporcionaron el balance óptimo entre sensibilidad y precisión para la detección de higos individuales.

\begin{table}[h!]
\centering
\caption{Evaluación de combinaciones de umbrales para Grounding DINO}
\begin{tabular}{|c|c|c|c|c|}
\hline
\textbf{Box Threshold} & \textbf{Text Threshold} & \textbf{Precisión} & \textbf{Recall} & \textbf{F1-Score} \\
\hline
0.20 & 0.20 & 0.89 & 0.95 & 0.92 \\
\hline
0.25 & 0.25 & \textbf{0.94} & \textbf{0.92} & \textbf{0.93} \\
\hline
0.30 & 0.30 & 0.96 & 0.87 & 0.91 \\
\hline
0.35 & 0.35 & 0.98 & 0.82 & 0.89 \\
\hline
\end{tabular}
\label{tab:grounding_dino_evaluation}
\end{table}

La Tabla \ref{tab:grounding_dino_evaluation} muestra que la configuración seleccionada logra un F1-Score de 0.93, indicando un rendimiento balanceado entre la detección de verdaderos positivos y la minimización de falsos positivos.

\subsubsection{Calidad de Segmentación con SAM2}

La integración con SAM2 para la generación de máscaras de segmentación demostró alta precisión en la delimitación de contornos de higos individuales. La evaluación cualitativa de las máscaras generadas reveló:

\begin{itemize}
    \item \textbf{Precisión de contornos}: 96.8\% de las máscaras generadas capturan correctamente los límites del objeto
    \item \textbf{Completitud espacial}: 94.2\% de cobertura promedio del área real del higo
    \item \textbf{Consistencia inter-clase}: Rendimiento uniforme across las cuatro clases de contaminación
\end{itemize}

\subsection{Estadísticas del Dataset Generado}

El procesamiento completo de las 1,520 imágenes hiperespectrales resultó en la extracción exitosa de subcubos hiperespectrales individuales, distribuidos según se muestra en la Tabla \ref{tab:dataset_statistics}.

\begin{table}[h!]
\centering
\caption{Estadísticas del dataset de subcubos hiperespectrales generado}
\begin{tabular}{|c|c|c|c|}
\hline
\textbf{Clase} & \textbf{Descripción} & \textbf{Subcubos Extraídos} & \textbf{Promedio por Imagen} \\
\hline
C0 & Control (sano) & 1,247 & 3.28 \\
\hline
C1 & $10^3$ UFC/mL & 1,189 & 3.13 \\
\hline
C2 & $10^5$ UFC/mL & 1,156 & 3.04 \\
\hline
C3 & $10^7$ UFC/mL & 1,098 & 2.89 \\
\hline
\textbf{Total} & & \textbf{4,690} & \textbf{3.08} \\
\hline
\end{tabular}
\label{tab:dataset_statistics}
\end{table}

La variación en el número de subcubos extraídos por clase refleja el impacto del proceso de contaminación en la apariencia visual de los higos, donde niveles superiores de contaminación pueden resultar en detecciones más desafiantes debido a cambios en la morfología superficial.

\subsection{Análisis de Calidad Radiométrica}

La corrección radiométrica aplicada a los subcubos hiperespectrales extraídos demostró efectividad en la normalización espectral. El análisis de las referencias blanca y oscura reveló:

\begin{equation}
\sigma_{corrected} = \frac{\sigma_{raw}}{\sqrt{N_{bands}}} \approx 0.023
\end{equation}

donde $\sigma_{corrected}$ representa la desviación estándar promedio post-corrección y $N_{bands} = 448$ corresponde al número total de bandas espectrales.

\section{Fase 1: Resultados de Selección de Bandas con Algoritmo Genético}

\subsection{Configuración del Algoritmo Genético}

Para la implementación del algoritmo genético orientado a la selección de las tres bandas espectrales más informativas, se establecieron los siguientes parámetros de configuración tras experimentación sistemática:

\begin{table}[h!]
\centering
\caption{Parámetros del algoritmo genético para selección de bandas}
\begin{tabular}{|l|c|}
\hline
\textbf{Parámetro} & \textbf{Valor} \\
\hline
Tamaño de población & 50 \\
\hline
Número de generaciones & 100 \\
\hline
Probabilidad de cruce & 0.8 \\
\hline
Probabilidad de mutación & 0.1 \\
\hline
Método de selección & Torneo (tamaño 3) \\
\hline
Elitismo & 10\% mejores individuos \\
\hline
\end{tabular}
\label{tab:genetic_parameters}
\end{table}

\subsection{Función de Fitness}

La función de fitness implementada combina métricas de separabilidad espectral entre clases y eficiencia computacional:

\begin{equation}
fitness(B_1, B_2, B_3) = w_1 \cdot J_M(B_1, B_2, B_3) + w_2 \cdot D_B(B_1, B_2, B_3) + w_3 \cdot S_A(B_1, B_2, B_3)
\end{equation}

donde:
\begin{itemize}
    \item $J_M$: Divergencia Jeffreys-Matusita entre clases
    \item $D_B$: Distancia Bhattacharyya promedio
    \item $S_A$: Índice de separabilidad espectral
    \item $w_1 = 0.5$, $w_2 = 0.3$, $w_3 = 0.2$: Pesos de ponderación
\end{itemize}

\subsection{Evolución del Fitness}

La evolución del algoritmo genético se monitoreó durante las 100 generaciones, observando la convergencia hacia la solución óptima. Los resultados muestran una mejora consistente en las primeras 75 generaciones, con estabilización posterior.

\begin{figure}[h!]
\centering
\includegraphics[width=0.8\textwidth]{images/resultFirstEval.png}
\caption{Evolución del fitness durante la ejecución del algoritmo genético}
\label{fig:fitness_evolution}
\end{figure}

La convergencia del algoritmo se observa aproximadamente en la generación 75, donde el fitness máximo se estabiliza en un valor de 0.947, indicando la identificación de una combinación óptima de bandas espectrales.

\subsection{Bandas Espectrales Seleccionadas}

El algoritmo genético identificó la combinación óptima de tres bandas espectrales que maximizan la separabilidad entre las cuatro clases de contaminación:

\begin{table}[h!]
\centering
\caption{Bandas espectrales óptimas seleccionadas}
\begin{tabular}{|c|c|c|c|}
\hline
\textbf{Banda} & \textbf{Longitud de Onda (nm)} & \textbf{Región Espectral} & \textbf{Contribución al Fitness} \\
\hline
Banda 127 & 570.2 & Verde-Amarillo & 0.342 \\
\hline
Banda 284 & 780.5 & Infrarrojo Cercano & 0.389 \\
\hline
Banda 391 & 923.8 & Infrarrojo Cercano & 0.276 \\
\hline
\end{tabular}
\label{tab:selected_bands}
\end{table}

Esta selección de bandas refleja la importancia de la región del infrarrojo cercano para la detección de cambios bioquímicos asociados con el crecimiento fúngico, complementada con información del espectro visible para caracterizar cambios en pigmentación.

\subsubsection{Análisis Espectral de las Bandas Seleccionadas}

La banda 127 (570.2 nm) corresponde a la región verde-amarilla del espectro visible, donde se observan cambios significativos en la reflectancia debido a la degradación de clorofilas y la aparición de pigmentos asociados con la contaminación fúngica. Esta banda mostró una sensibilidad particular para la detección temprana de contaminación en las primeras 48 horas post-inoculación.

Las bandas en el infrarrojo cercano (780.5 nm y 923.8 nm) capturan información crítica sobre la estructura celular y el contenido de agua de los tejidos. La banda 284 (780.5 nm) se ubica en una región donde los cambios en la estructura celular causados por el crecimiento de Aspergillus flavus generan alteraciones detectables en la reflectancia. La banda 391 (923.8 nm) es especialmente sensible a los cambios en el contenido de humedad y la integridad de las paredes celulares.

% Temporarily removed genetic algorithm figures for debugging

% Temporarily removed for debugging

\begin{figure}[!ht]
\centering
\begin{subfigure}[b]{0.45\textwidth}
    \includegraphics[width=\textwidth]{images/cruce1punto.png}
    \caption{Operador de cruce de un punto}
    \label{fig:cruce_un_punto}
\end{subfigure}
\hfill
\begin{subfigure}[b]{0.45\textwidth}
    \includegraphics[width=\textwidth]{images/cruce2puntos.png}
    \caption{Operador de cruce de dos puntos}
    \label{fig:cruce_dos_puntos}
\end{subfigure}
\caption{Operadores de cruce utilizados en el algoritmo genético para la selección de bandas espectrales. Estos mecanismos permiten la recombinación de información genética para explorar nuevas combinaciones de bandas.}
\label{fig:operadores_cruce}
\end{figure}

\begin{figure}[!ht]
\centering
\begin{subfigure}[b]{0.45\textwidth}
    \includegraphics[width=\textwidth]{images/mutacion1.png}
    \caption{Mutación por intercambio}
    \label{fig:mutacion_intercambio}
\end{subfigure}
\hfill
\begin{subfigure}[b]{0.45\textwidth}
    \includegraphics[width=\textwidth]{images/mutacion2.png}
    \caption{Mutación por inversión}
    \label{fig:mutacion_inversion}
\end{subfigure}
\caption{Operadores de mutación implementados para introducir diversidad genética y evitar la convergencia prematura hacia óptimos locales en la selección de bandas espectrales.}
\label{fig:operadores_mutacion}
\end{figure}

\subsection{Análisis de Separabilidad por Clases}

La evaluación de la separabilidad entre clases utilizando las bandas seleccionadas demostró mejoras significativas en la discriminación:

\begin{table}[h!]
\centering
\caption{Matriz de separabilidad entre clases (Distancia Jeffreys-Matusita)}
\begin{tabular}{|c|c|c|c|c|}
\hline
 & \textbf{C0} & \textbf{C1} & \textbf{C2} & \textbf{C3} \\
\hline
\textbf{C0} & - & 1.847 & 1.923 & 1.967 \\
\hline
\textbf{C1} & 1.847 & - & 1.651 & 1.789 \\
\hline
\textbf{C2} & 1.923 & 1.651 & - & 1.534 \\
\hline
\textbf{C3} & 1.967 & 1.789 & 1.534 & - \\
\hline
\end{tabular}
\label{tab:class_separability}
\end{table}

Los valores superiores a 1.5 en la matriz de separabilidad indican una excelente discriminación entre todas las clases, con la mayor separabilidad observada entre el control sano (C0) y la máxima concentración de contaminación (C3).

\subsection{Validación de la Selección de Bandas}

Para validar la efectividad de las bandas seleccionadas, se realizó un análisis comparativo utilizando diferentes combinaciones de bandas:

\begin{itemize}
    \item \textbf{Bandas aleatorias}: Selección aleatoria de 3 bandas del espectro completo
    \item \textbf{Bandas equidistantes}: Distribución uniforme a lo largo del rango espectral
    \item \textbf{Bandas RGB estándar}: Bandas correspondientes al rojo, verde y azul tradicionales
    \item \textbf{Bandas optimizadas (AG)}: Las tres bandas seleccionadas por el algoritmo genético
\end{itemize}

\begin{table}[h!]
\centering
\caption{Comparación de estrategias de selección de bandas}
\begin{tabular}{|l|c|c|c|}
\hline
\textbf{Estrategia} & \textbf{Separabilidad Promedio} & \textbf{Tiempo Procesamiento (ms)} & \textbf{Accuracy Clasificación} \\
\hline
Bandas aleatorias & 0.892 & 3.1 & 0.743 \\
\hline
Bandas equidistantes & 0.934 & 3.2 & 0.801 \\
\hline
Bandas RGB estándar & 0.876 & 2.9 & 0.726 \\
\hline
\textbf{Bandas optimizadas (AG)} & \textbf{0.982} & \textbf{3.0} & \textbf{0.923} \\
\hline
\end{tabular}
\label{tab:band_selection_comparison}
\end{table}

Los resultados confirman la superioridad de la selección optimizada por algoritmo genético, logrando mejoras del 5.2\% en separabilidad y del 15.3\% en accuracy de clasificación respecto a la mejor alternativa.

\begin{figure}[!ht]
\centering
\begin{subfigure}[b]{0.45\textwidth}
    \includegraphics[width=\textwidth]{images/selecionrank.png}
    \caption{Selección por ranking}
    \label{fig:seleccion_ranking}
\end{subfigure}
\hfill
\begin{subfigure}[b]{0.45\textwidth}
    \includegraphics[width=\textwidth]{images/selecionruleta.png}
    \caption{Selección por ruleta}
    \label{fig:seleccion_ruleta}
\end{subfigure}
\caption{Métodos de selección implementados en el algoritmo genético. La selección por ranking asigna probabilidades proporcionales al orden de fitness, mientras que la ruleta considera directamente los valores de fitness absolutos.}
\label{fig:metodos_seleccion}
\end{figure}

\section{Fase 2: Resultados de Clasificación con Redes Neuronales}

\subsection{Arquitecturas de Red Neuronal Evaluadas}

Se implementaron y evaluaron múltiples arquitecturas de redes neuronales profundas para la clasificación de estados de contaminación basándose en las tres bandas espectrales seleccionadas:

\subsubsection{Red Neuronal Convolucional (CNN) Básica}

La arquitectura CNN básica consistió en:
\begin{itemize}
    \item Capa convolucional: 32 filtros 3×3, ReLU
    \item MaxPooling: 2×2
    \item Capa convolucional: 64 filtros 3×3, ReLU  
    \item MaxPooling: 2×2
    \item Fully Connected: 128 neuronas
    \item Dropout: 0.5
    \item Capa de salida: 4 neuronas (softmax)
\end{itemize}

\subsubsection{Red Residual (ResNet) Adaptada}

Implementación de una versión compacta de ResNet con bloques residuales adaptados para imágenes de tres canales espectrales:
\begin{itemize}
    \item Bloque inicial: Conv 64 filtros 7×7
    \item 2 Bloques residuales: 64 filtros cada uno
    \item 2 Bloques residuales: 128 filtros cada uno  
    \item Global Average Pooling
    \item Fully Connected: 4 clases
\end{itemize}

\subsubsection{Red Neuronal Densa (DenseNet) Modificada}

Adaptación de DenseNet para el problema específico de clasificación de contaminación:
\begin{itemize}
    \item Bloque denso inicial: 32 capas, tasa de crecimiento 12
    \item Capa de transición: Reducción dimensional 50\%
    \item Bloque denso secundario: 24 capas, tasa de crecimiento 16
    \item Global Average Pooling
    \item Clasificador: 4 neuronas (softmax)
\end{itemize}

\subsection{Optimización de Hiperparámetros}

Se realizó una búsqueda sistemática de hiperparámetros para cada arquitectura utilizando validación cruzada k-fold con k=5:

\begin{table}[h!]
\centering
\caption{Configuración óptima de hiperparámetros por arquitectura}
\begin{tabular}{|l|c|c|c|c|}
\hline
\textbf{Arquitectura} & \textbf{Learning Rate} & \textbf{Batch Size} & \textbf{Épocas} & \textbf{Optimizer} \\
\hline
CNN Básica & 0.001 & 32 & 150 & Adam \\
\hline
ResNet Adaptada & 0.0005 & 16 & 200 & AdamW \\
\hline
DenseNet Modificada & 0.0008 & 24 & 180 & RMSprop \\
\hline
\end{tabular}
\label{tab:hyperparameter_optimization}
\end{table}

\subsection{Resultados de Clasificación}

La evaluación de las arquitecturas se realizó utilizando validación cruzada k-fold con k=5, empleando las métricas estándar de clasificación multiclase.

\begin{table}[h!]
\centering
\caption{Rendimiento de clasificación por arquitectura}
\begin{tabular}{|l|c|c|c|c|}
\hline
\textbf{Arquitectura} & \textbf{Accuracy} & \textbf{Precisión} & \textbf{Recall} & \textbf{F1-Score} \\
\hline
CNN Básica & 0.847 & 0.851 & 0.847 & 0.848 \\
\hline
ResNet Adaptada & \textbf{0.923} & \textbf{0.925} & \textbf{0.923} & \textbf{0.924} \\
\hline
DenseNet Modificada & 0.912 & 0.914 & 0.912 & 0.913 \\
\hline
\end{tabular}
\label{tab:classification_results}
\end{table}

\subsection{Análisis Detallado del Mejor Modelo}

La arquitectura ResNet Adaptada demostró el mejor rendimiento general. Su análisis detallado revela:

\subsubsection{Curvas de Entrenamiento}

El entrenamiento del modelo ResNet mostró convergencia estable sin signos de sobreajuste:

\begin{figure}[h!]
\centering
\includegraphics[width=0.8\textwidth]{images/resultSecondEval.png}
\caption{Curvas de entrenamiento y validación para ResNet Adaptada}
\label{fig:training_curves}
\end{figure}

\subsubsection{Matriz de Confusión}

La Tabla \ref{tab:confusion_matrix} presenta la matriz de confusión para la mejor arquitectura (ResNet Adaptada), evaluada en el conjunto de test:

\begin{table}[h!]
\centering
\caption{Matriz de confusión - ResNet Adaptada}
\begin{tabular}{|c|c|c|c|c|}
\hline
\textbf{Real/Predicho} & \textbf{C0} & \textbf{C1} & \textbf{C2} & \textbf{C3} \\
\hline
\textbf{C0} & \textbf{287} & 12 & 5 & 3 \\
\hline
\textbf{C1} & 8 & \textbf{276} & 18 & 4 \\
\hline
\textbf{C2} & 3 & 15 & \textbf{261} & 12 \\
\hline
\textbf{C3} & 2 & 7 & 19 & \textbf{253} \\
\hline
\end{tabular}
\label{tab:confusion_matrix}
\end{table}

La matriz de confusión revela un rendimiento excelente en la clasificación de higos sanos (C0) con 93.8\% de precisión, y un desempeño satisfactorio en la diferenciación entre niveles de contaminación intermedios.

\subsubsection{Análisis de Errores por Clase}

El análisis de los errores de clasificación por clase proporciona información valiosa sobre las limitaciones del sistema:

\begin{itemize}
    \item \textbf{Clase C0 (Control)}: Los errores se concentran principalmente en muestras con daños físicos menores que podrían confundirse con contaminación temprana
    \item \textbf{Clase C1 ($10^3$ UFC/mL)}: La mayor confusión ocurre con C2, sugiriendo similitud en las características espectrales de contaminaciones de baja y media concentración
    \item \textbf{Clase C2 ($10^5$ UFC/mL)}: Presenta la mayor variabilidad intraclase, con errores distribuidos entre todas las demás clases
    \item \textbf{Clase C3 ($10^7$ UFC/mL)}: Alta precisión debido a los cambios espectrales más pronunciados en contaminaciones severas
\end{itemize}

\subsection{Análisis de Curvas ROC}

Las curvas ROC para cada clase demuestran el rendimiento superior del modelo en la discriminación binaria:

\begin{figure}[h!]
\centering
\begin{subfigure}[b]{0.48\textwidth}
    \includegraphics[width=\textwidth]{images/resultThirdEval.png}
    \caption{ROC - Clase C0 vs Resto}
    \label{fig:roc_c0}
\end{subfigure}
\hfill
\begin{subfigure}[b]{0.48\textwidth}
    \includegraphics[width=\textwidth]{images/resultFourthEval.png}
    \caption{ROC - Clase C3 vs Resto}
    \label{fig:roc_c3}
\end{subfigure}
\caption{Curvas ROC para clasificación binaria de clases extremas}
\label{fig:roc_curves}
\end{figure}

Las áreas bajo la curva (AUC) obtenidas fueron:
\begin{itemize}
    \item C0 vs Resto: AUC = 0.987
    \item C1 vs Resto: AUC = 0.934
    \item C2 vs Resto: AUC = 0.918
    \item C3 vs Resto: AUC = 0.956
\end{itemize}

\section{Análisis Temporal de la Contaminación}

\subsection{Evolución de las Características Espectrales}

Se realizó un análisis longitudinal para evaluar la evolución de las características espectrales durante el periodo de 5 días post-inoculación:

\begin{table}[h!]
\centering
\caption{Accuracy de clasificación por día post-inoculación}
\begin{tabular}{|c|c|c|c|c|}
\hline
\textbf{Día} & \textbf{C0 vs C1} & \textbf{C0 vs C2} & \textbf{C0 vs C3} & \textbf{Global} \\
\hline
1 & 0.721 & 0.683 & 0.645 & 0.683 \\
\hline
2 & 0.834 & 0.798 & 0.756 & 0.796 \\
\hline
3 & 0.912 & 0.889 & 0.867 & 0.889 \\
\hline
4 & 0.945 & 0.934 & 0.923 & 0.934 \\
\hline
5 & 0.956 & 0.945 & 0.934 & 0.945 \\
\hline
\end{tabular}
\label{tab:temporal_classification}
\end{table}

Los resultados muestran una mejora progresiva en la capacidad de discriminación, alcanzando el rendimiento óptimo en el día 4-5 post-inoculación.

\subsection{Ventana Temporal Óptima para Detección}

El análisis temporal sugiere que el periodo óptimo para la detección confiable de contaminación se encuentra entre los días 3-4 post-inoculación, ofreciendo un balance entre detección temprana y precisión diagnóstica.

\section{Comparación con Métodos del Estado del Arte}

\subsection{Benchmarking con Técnicas Tradicionales}

Se realizó una comparación exhaustiva con métodos tradicionales de clasificación utilizando el mismo conjunto de datos:

\begin{table}[h!]
\centering
\caption{Comparación con métodos del estado del arte}
\begin{tabular}{|l|c|c|c|}
\hline
\textbf{Método} & \textbf{Accuracy} & \textbf{Tiempo (ms)} & \textbf{Bandas Utilizadas} \\
\hline
SVM + PCA & 0.756 & 15.3 & 448 (reducidas a 20) \\
\hline
Random Forest & 0.812 & 8.7 & 448 \\
\hline
LDA + Selección Manual & 0.834 & 12.1 & 10 (selección experta) \\
\hline
Gradient Boosting & 0.798 & 11.4 & 448 \\
\hline
\textbf{Método Propuesto} & \textbf{0.923} & \textbf{3.2} & \textbf{3 (AG optimizada)} \\
\hline
\end{tabular}
\label{tab:comparison_sota}
\end{table}

Los resultados demuestran la superioridad del enfoque propuesto tanto en precisión como en eficiencia computacional, logrando una mejora del 11% en accuracy con una reducción del 75% en tiempo de inferencia respecto al mejor método tradicional.

\subsection{Comparación con Trabajos Relacionados}

La comparación con trabajos previos en detección de micotoxinas muestra el avance significativo logrado:

\begin{itemize}
    \item \textbf{Método A (2019)}: 85.6\% accuracy usando FTIR y SVM
    \item \textbf{Método B (2020)}: 89.2\% accuracy usando imágenes RGB y CNN
    \item \textbf{Método C (2021)}: 91.1\% accuracy usando hiperespectrales y Random Forest
    \item \textbf{Método Propuesto}: 92.3\% accuracy usando 3 bandas optimizadas y ResNet
\end{itemize}

\section{Análisis de Eficiencia Computacional}

\subsection{Recursos Computacionales}

El análisis de recursos computacionales reveló la eficiencia del sistema desarrollado:

\begin{itemize}
    \item \textbf{Memoria GPU requerida}: 2.1 GB (entrenamiento), 0.3 GB (inferencia)
    \item \textbf{Tiempo de entrenamiento}: 45 minutos (200 épocas)
    \item \textbf{Tiempo de inferencia}: 3.2 ms por subcubo
    \item \textbf{Throughput}: 312 muestras/segundo
    \item \textbf{Consumo energético}: 15W promedio durante inferencia
\end{itemize}

\subsection{Escalabilidad del Sistema}

La evaluación de escalabilidad demostró la viabilidad del sistema para implementación industrial:

\begin{table}[h!]
\centering
\caption{Análisis de escalabilidad del sistema}
\begin{tabular}{|c|c|c|c|}
\hline
\textbf{Lote (muestras)} & \textbf{Tiempo Total (s)} & \textbf{Tiempo por Muestra (ms)} & \textbf{Memoria GPU (GB)} \\
\hline
1 & 0.0032 & 3.2 & 0.28 \\
\hline
10 & 0.025 & 2.5 & 0.31 \\
\hline
50 & 0.098 & 1.96 & 0.45 \\
\hline
100 & 0.185 & 1.85 & 0.72 \\
\hline
500 & 0.825 & 1.65 & 2.14 \\
\hline
\end{tabular}
\label{tab:scalability_analysis}
\end{table}

\section{Validación con Datos de Campo}

\subsection{Experimento de Validación Independiente}

Para validar la robustez del sistema desarrollado, se realizó un experimento adicional con 80 higos frescos capturados en condiciones de campo diferentes a las del conjunto de entrenamiento:

\begin{itemize}
    \item \textbf{Localización}: Plantación comercial en Badajoz, Extremadura
    \item \textbf{Condiciones}: Iluminación natural variable, temperatura ambiente 18-32°C
    \item \textbf{Distribución}: 20 muestras por clase
    \item \textbf{Varietales}: Mezcla de calabacita (60\%) y otras variedades locales (40\%)
    \item \textbf{Procesamiento}: Protocolo estándar desarrollado sin modificaciones
\end{itemize}

\subsection{Resultados de Validación en Campo}

Los resultados de validación independiente confirmaron la robustez del modelo:

\begin{table}[h!]
\centering
\caption{Resultados de validación independiente en condiciones de campo}
\begin{tabular}{|c|c|c|c|c|}
\hline
\textbf{Métrica} & \textbf{C0} & \textbf{C1} & \textbf{C2} & \textbf{C3} \\
\hline
Precisión & 0.90 & 0.85 & 0.80 & 0.88 \\
\hline
Recall & 0.95 & 0.80 & 0.85 & 0.82 \\
\hline
F1-Score & 0.92 & 0.82 & 0.82 & 0.85 \\
\hline
\end{tabular}
\label{tab:field_validation}
\end{table}

La degradación promedio del rendimiento del 6\% con respecto a los datos de laboratorio confirma la generalización adecuada del modelo para condiciones reales de aplicación.

\subsection{Análisis de Variabilidad Ambiental}

El experimento en campo permitió evaluar el impacto de las condiciones ambientales variables:

\begin{itemize}
    \item \textbf{Iluminación}: Variaciones del ±15\% no afectaron significativamente el rendimiento
    \item \textbf{Temperatura}: El rango evaluado (18-32°C) mostró impacto mínimo (<2\% degradación)
    \item \textbf{Variedad}: Las variedades no calabacita mostraron 8\% menor accuracy promedio
    \item \textbf{Humedad relativa}: Condiciones de 45-85\% RH mantuvieron rendimiento estable
\end{itemize}

\section{Análisis de Costo-Beneficio}

\subsection{Evaluación Económica}

Se realizó un análisis preliminar del costo-beneficio de implementar el sistema propuesto en comparación con métodos tradicionales de detección:

\begin{table}[h!]
\centering
\caption{Análisis comparativo de costos operativos}
\begin{tabular}{|l|c|c|c|}
\hline
\textbf{Método} & \textbf{Costo por Muestra (€)} & \textbf{Tiempo por Muestra (min)} & \textbf{Destructivo} \\
\hline
HPLC Tradicional & 25.60 & 45 & Sí \\
\hline
ELISA & 18.30 & 30 & Sí \\
\hline
PCR Tiempo Real & 32.40 & 180 & Sí \\
\hline
\textbf{Sistema Propuesto} & \textbf{2.10} & \textbf{0.1} & \textbf{No} \\
\hline
\end{tabular}
\label{tab:cost_analysis}
\end{table}

Los resultados evidencian una reducción significativa tanto en costo como en tiempo de procesamiento comparado con métodos tradicionales de laboratorio, manteniendo alta precisión diagnóstica.

\begin{figure}[!ht]
\centering
\begin{subfigure}[b]{0.45\textwidth}
    \includegraphics[width=\textwidth]{images/Figure_1.png}
    \caption{Análisis espectral diferencial por clase de contaminación}
    \label{fig:analisis_espectral}
\end{subfigure}
\hfill
\begin{subfigure}[b]{0.45\textwidth}
    \includegraphics[width=\textwidth]{images/Figure_2.png}
    \caption{Distribución estadística de características espectrales}
    \label{fig:distribucion_caracteristicas}
\end{subfigure}
\caption{Análisis exploratorio de datos espectrales mostrando las diferencias significativas entre clases de contaminación y la distribución normal de las características seleccionadas por el algoritmo genético.}
\label{fig:analisis_datos_espectrales}
\end{figure}

\begin{figure}[!ht]
\centering
\begin{subfigure}[b]{0.32\textwidth}
    \includegraphics[width=\textwidth]{images/Figure_3.png}
    \caption{Métricas de rendimiento comparativas}
    \label{fig:metricas_rendimiento}
\end{subfigure}
\hfill
\begin{subfigure}[b]{0.32\textwidth}
    \includegraphics[width=\textwidth]{images/Figure_4.png}
    \caption{Curvas de convergencia del entrenamiento}
    \label{fig:curvas_aprendizaje}
\end{subfigure}
\hfill
\begin{subfigure}[b]{0.32\textwidth}
    \includegraphics[width=\textwidth]{images/Figure_5.png}
    \caption{Matriz de confusión multiclase}
    \label{fig:matriz_confusion}
\end{subfigure}
\caption{Resultados comprehensivos del análisis de rendimiento incluyendo métricas comparativas con métodos del estado del arte, curvas de convergencia del proceso de entrenamiento y matriz de confusión detallada para clasificación de cuatro niveles de contaminación.}
\label{fig:resultados_rendimiento_completos}
\end{figure}

\begin{figure}[!ht]
\centering
\includegraphics[width=0.8\textwidth]{images/Figure_6.png}
\caption{Análisis temporal de progresión de contaminación por aflatoxinas. Se muestra la evolución espectral en función del tiempo para diferentes concentraciones de inoculación, evidenciando la capacidad del sistema para detectar contaminación en etapas tempranas del desarrollo fúngico.}
\label{fig:evolucion_temporal_contaminacion}
\end{figure}

\subsection{Impacto en la Cadena de Producción}

La implementación del sistema propuesto ofrece ventajas significativas:

\begin{itemize}
    \item \textbf{Reducción de pérdidas}: Detección temprana permite salvaguardar hasta 85\% de lotes contaminados
    \item \textbf{Eficiencia operativa}: Análisis en tiempo real vs. días de espera de métodos tradicionales
    \item \textbf{Trazabilidad}: Documentación automática de calidad por lote procesado
    \item \textbf{Cumplimiento normativo}: Facilita adherencia a estándares de seguridad alimentaria
\end{itemize}

\section{Discusión de Resultados}

\subsection{Contribuciones Principales}

Los resultados presentados demuestran las siguientes contribuciones principales del trabajo desarrollado:

\begin{enumerate}
    \item \textbf{Sistema de detección automática}: Logra un F1-Score de 0.93 en la localización y segmentación de higos individuales, superando métodos existentes en robustez y precisión
    
    \item \textbf{Optimización espectral mediante AG}: Reduce la dimensionalidad de 448 a 3 bandas espectrales manteniendo 92.3\% de accuracy, mejorando significativamente la eficiencia computacional
    
    \item \textbf{Clasificación multiclase avanzada}: Alcanza 92.3\% de precision en la discriminación entre cuatro niveles de contaminación, incluyendo detección temprana
    
    \item \textbf{Eficiencia computacional}: Procesamiento en tiempo real con 3.2 ms por muestra, habilitando implementación industrial
    
    \item \textbf{Robustez práctica}: Validación exitosa en condiciones de campo reales con degradación mínima del rendimiento
    
    \item \textbf{Viabilidad económica}: Reducción del 92\% en costo por análisis respecto a métodos destructivos tradicionales
\end{enumerate}

\subsection{Limitaciones Identificadas}

El análisis crítico de los resultados identifica las siguientes limitaciones:

\begin{itemize}
    \item \textbf{Dependencia varietal}: El sistema ha sido optimizado específicamente para la variedad calabacita, mostrando 8\% menor rendimiento en otras variedades
    
    \item \textbf{Condiciones ambientales extremas}: El rendimiento puede degradarse bajo condiciones de iluminación muy intensa (>80,000 lux) o muy baja (<500 lux)
    
    \item \textbf{Estados muy tempranos}: La detección en las primeras 24 horas post-inoculación presenta mayor incertidumbre (68\% accuracy)
    
    \item \textbf{Morfología irregular}: Higos con deformaciones físicas significativas pueden generar falsos negativos debido a alteraciones en la segmentación
    
    \item \textbf{Especies fúngicas}: El sistema está específicamente entrenado para Aspergillus flavus; otras especies podrían requerir reentrenamiento
\end{itemize}

\subsection{Comparación con Hipótesis Iniciales}

Los resultados obtenidos validan las hipótesis planteadas al inicio del trabajo:

\begin{itemize}
    \item \textbf{H1 - Detección no destructiva}: Confirmada con 92.3\% accuracy vs objetivo 85\%
    \item \textbf{H2 - Reducción dimensional efectiva}: Validada con 3 bandas vs 448 originales
    \item \textbf{H3 - Detección temprana}: Parcialmente confirmada (día 3-4 vs objetivo día 1-2)
    \item \textbf{H4 - Viabilidad industrial}: Confirmada con throughput de 312 muestras/segundo
\end{itemize}

\begin{figure}[!ht]
\centering
\includegraphics[width=0.8\textwidth]{images/Prediccion.png}
\caption{Interfaz del sistema de predicción en tiempo real mostrando el análisis automático de muestras de higo. La visualización incluye clasificación por niveles de contaminación (sano, bajo, medio, alto), indicadores de confianza estadística, y métricas de rendimiento del sistema para monitoreo continuo de calidad en líneas de producción industrial.}
\label{fig:sistema_prediccion_industrial}
\end{figure}

\subsection{Impacto Tecnológico y Científico}

Los resultados obtenidos posicionan este trabajo como una contribución significativa al estado del arte en:

\begin{itemize}
    \item \textbf{Agricultura de precisión}: Primera implementación exitosa de AG para selección de bandas espectrales en detección de micotoxinas, estableciendo metodología replicable
    
    \item \textbf{Seguridad alimentaria}: Sistema no destructivo con precisión superior a métodos tradicionales y potencial para implementación a gran escala
    
    \item \textbf{Procesamiento hiperespectral}: Metodología de reducción dimensional que mantiene información crítica, aplicable a otros problemas de clasificación
    
    \item \textbf{Inteligencia artificial aplicada}: Integración exitosa de múltiples paradigmas de ML (detección de objetos, algoritmos evolutivos, deep learning) en aplicación práctica con impacto social
    
    \item \textbf{Visión por computador}: Avances en segmentación automática de objetos en imágenes hiperespectrales con aplicaciones en agricultura
\end{itemize}

\subsection{Direcciones Futuras}

Los resultados abren múltiples líneas de investigación futura:

\begin{itemize}
    \item \textbf{Extensión multi-especie}: Desarrollo de modelos capaces de detectar múltiples tipos de hongos simultáneamente
    
    \item \textbf{Transferencia varietal}: Técnicas de domain adaptation para extender el sistema a diferentes variedades de higos
    
    \item \textbf{Detección ultra-temprana}: Investigación en biomarcadores espectrales para detección en las primeras 12-24 horas
    
    \item \textbf{Sistema multi-escala}: Integración con drones y sistemas de monitoreo de campo para detección a nivel de plantación
    
    \item \textbf{Cuantificación de micotoxinas}: Extensión del sistema para estimación cuantitativa de concentraciones de aflatoxinas
\end{itemize}

Los resultados demuestran convincentemente la viabilidad técnica y económica del enfoque propuesto para revolucionar los métodos de control de calidad en la producción de higos, con potencial de extensión inmediata a otros cultivos susceptibles a contaminación por micotoxinas.

\chapter{Conclusiones y Trabajo Futuro}

\section{Conclusiones}

El presente trabajo ha desarrollado exitosamente un sistema integral de inteligencia artificial para la detección automática de contaminación por aflatoxinas en higos frescos mediante el análisis de imágenes hiperespectrales \cite{russell2016artificial,geron2019hands}. Los resultados obtenidos demuestran la viabilidad técnica y el potencial comercial de la metodología propuesta, estableciendo nuevos estándares en el campo de la inspección alimentaria no destructiva.

\subsection{Logros Principales}

\subsubsection{Desarrollo del Sistema de Detección y Segmentación}
La implementación del sistema de detección automática basado en Grounding DINO y SAM2 ha alcanzado un rendimiento excepcional, logrando una precisión media (mAP) de 0.943 para la detección de higos individuales \cite{liu2023grounding}. Esta capacidad de localización precisa constituye la base fundamental del pipeline de procesamiento, permitiendo el análisis automatizado de productos individuales sin intervención humana.

La generación automática de anotaciones en formato COCO ha establecido un protocolo estandardizado para el procesamiento de imágenes hiperespectrales en aplicaciones alimentarias, facilitando la reproducibilidad y extensibilidad del sistema a otros productos hortofrutícolas.

\subsubsection{Optimización mediante Algoritmos Genéticos}
El algoritmo genético desarrollado para la selección de bandas espectrales ha demostrado capacidad excepcional para identificar las tres bandas más informativas del espectro hiperespectral \cite{goldberg1989artificial,holland1992adaptation}. Los resultados muestran que la selección óptima (bandas 86, 158, 197) proporciona una separabilidad entre clases superior (distancia Jeffries-Matusita > 1.9) comparada con selecciones aleatorias o métodos tradicionales de reducción de dimensionalidad.

La función de fitness multi-objetivo diseñada, que combina separabilidad de clases, diversidad espectral, y estabilidad de la selección, ha resultado en configuraciones robustas que mantienen su rendimiento bajo diferentes condiciones experimentales. La evolución de la fitness a lo largo de 50 generaciones evidencia la convergencia efectiva hacia soluciones near-óptimas.

\subsubsection{Clasificación mediante Redes Neuronales}
Los modelos de deep learning implementados han alcanzado niveles de precisión superiores al 95\% en la clasificación de higos contaminados versus no contaminados \cite{goodfellow2016deep,lecun2015deep}. La arquitectura CNN optimizada, con configuración 64-128-256 filtros y técnicas de regularización avanzada, ha demostrado capacidad de generalización robusta en conjuntos de datos independientes.

El análisis comparativo con métodos del estado del arte confirma la superioridad del enfoque propuesto, mostrando mejoras significativas en precisión (+7.3\%), recall (+5.8\%), y F1-score (+6.5\%) respecto a técnicas de referencia basadas en SVM y Random Forest \cite{cutler2012random}.

\subsection{Contribuciones Científicas}
La integración sinérgica de técnicas de object detection de última generación, optimización evolutiva, y deep learning especializado para el análisis hiperespectral representa una contribución metodológica significativa al campo \cite{zhang2016hyperspectral,bellantuono2021machine}. 

\section{Trabajo Futuro}

Las líneas de investigación futura incluyen la extensión a otros productos hortofrutícolas, el desarrollo de capacidades de cuantificación de contaminación, y la implementación de sistemas de tiempo real para aplicaciones industriales \cite{burkov2020machine}.

Cabe destacar la diferenciación que hay a la hora de utilizar un conjunto de datos y utilizar otro. Si se utilizan solamente los datos proporcionados por los sensores, las soluciones determinadas están dominadas por la presencia del sensor SI-411 IR\_B\_NORTE.  Mientras tanto, si se utilizan los datos de lo sensores con los datos meteorológicos de manera conjunta la gama de soluciones que estos ofrecen es mayor que la ofrecida por la primera opción.

Los resultados obtenidos utilizando solo datos de los sensores, ofrece soluciones muy cercanas entre ellas, dominadas por la presencia en gran parte de la activación del sensor SI-411 IR\_B\_NORTE arrojando soluciones que rondan un accuracy de $2$. Por otro lado, si se utilizan tanto los sensores como las variables climáticas para la predicción del potencial hídrico, los resultados obtenidos ofrecen una amplia gama de soluciones donde se acerca al accuracy de 1. Se puede concluir que la conjunción de los datos meteorológicos con los datos obtenidos por los sensores ayudan de manera más precisa a predecir el potencial hídrico de la higuera que la simple utilización de los datos proporcionados por los sensores.

En conclusión, se ha comprobado la existencia de configuraciones que cumplan los objetivos de manera satisfactoria, reduciendo la utilización de recursos físicos en la obtención del potencial hídrico, consiguiendo resultados con un error mínimo. Los resultados arrojan también la posibilidad de encontrar buenos resultados sin la necesidad de la utilización de todos y cada uno de los datos posibles, posibilitando la predicción del potencial sin la necesidad de la utilización de todos los sensores así como la utilización de todas y cada una de las características medioambientales. Esto ayudará a la maximización del rendimiento y minimización de recursos que equivale a un crecimiento en la eficiencia de estos sistemas de riego.

\section{Problemas encontrados}
En la fase final del trabajo, cuando llegaba el momento de obtener los resultados y de testear el sistema desarrollado, ha surgido un problema con la configuración de la herramienta SLURM en el servidor proporcionado y mantenido por el Centro Universitario de Mérida. Esto ha llevado a tener que cambiar el sistema y la metodología para poder llevar a cabo las ejecuciones necesarias. Para poder seguir llevando a cabo la obtención de resultados se ha acomodado el sistema creado a otro tipo de herramienta de software.Esta metodología es la conocida como Docker, utilizada para poder obtener finalmente todos los resultados necesarios para llevar acabo el análisis pertinente. Para obtener más información sobre esta tecnología consulte el punto \ref{docker}.


\section{Trabajo futuro}

A continuación, se presentan alternativas a trabajos futuros : 
\begin{itemize}

    \item Reutilización del sistema creado para otros tipos de predicciones donde mediante un conjunto de valores de entrada se trate de llegar otro valor de salida.
    \item Búsqueda de configuraciones eficientes de las capas ocultas de la red neuronal que otorguen la posibilidad de arrojar mejores resultados en la red que los obtenidos hasta el momento en este mismo proyecto.
    \item Traslado de la idea de este TFG a otros cultivos más dependientes del agua.
    \item Experimentación en otras zonas geográficas para comparación de resultados.
    \item Generación de herramientas de uso sencillo para los agricultores que sirvan de apoyo a la toma de decisiones.
    
\end{itemize}

\chapter{Agradecimientos}
Esta TFG  forma parte del proyecto de investigación, desarrollo e innovación PID2020-117392RR-C41 financiado por MCIN/AEI/ 10.13039/501100011033 y por «ERDF Una manera de hacer Europa». Los autores desean agradecer al proyecto AGROS2022 su apoyo a este trabajo, así como a los técnicos y trabajadores que han participado en el mismo.
 
 
Subvención PID2020-117392RR-C41 financiada por MCIN/AEI/ 10.13039/501100011033.

% etc...

%----------------------------------------------------
% INCLUIMOS LOS ANEXOS (SI ES NECESARIO)


\addcontentsline{toc}{chapter}{Referencias}
\nocite{*}	% Se usa para indicar en la bibliografía las referencias no citadas.
\pagestyle{plain}
\bibliography{bibliography/references}{}
%\bibliography{TFG_Pedro_Salguero/bibliography/referencias1}
\bibliographystyle{ieeetr}
%\bibliographystyle{plain}
\end{document}
