\renewcommand{\keywords}{{\itshape \bfseries Palabras clave - }}
\chapter*{Resumen} 
%\addcontentsline{toc}{chapter}{Resumen} % si queremos que aparezca en el índice

La seguridad alimentaria representa uno de los desafíos más críticos de la actualidad, especialmente en el contexto de la detección temprana de contaminantes que pueden representar riesgos significativos para la salud humana. Las micotoxinas, sustancias tóxicas producidas por hongos como el Aspergillus flavus, constituyen una amenaza considerable en la cadena alimentaria, siendo clasificadas por la Agencia Internacional para la Investigación del Cáncer como sustancias carcinógenas del grupo 1.

\vspace{5mm}

La detección tradicional de micotoxinas en productos agrícolas requiere métodos invasivos que implican la destrucción de las muestras, lo que resulta en pérdidas económicas significativas y limitaciones en el control de calidad durante el proceso productivo. En este contexto, el desarrollo de técnicas no invasivas para la detección temprana de contaminación por aflatoxinas se presenta como una necesidad imperante para la industria agroalimentaria.

\vspace{5mm}

Este Trabajo Fin de Grado se centra en el desarrollo de un sistema de inteligencia artificial capaz de detectar la contaminación por micotoxinas en higos frescos mediante el análisis de imágenes hiperespectrales y técnicas de inteligencia artificial. El proyecto aborda específicamente la clasificación de estados de enfermedad en frutos de higuera, utilizando un algoritmo genético junto a una red neuronal para identificar la presencia de Aspergillus flavus en diferentes concentraciones de contaminación.

\vspace{5mm}

La metodología propuesta se estructura en múltiples fases, comenzando con la localización y segmentación de higos individuales mediante detección RGB, seguida de la selección de bandas espectrales más informativas utilizando un algoritmo genético, y culminando con el procesamiento de parches espectrales completos mediante transformadas wavelet. Esta aproximación multimodal permite aprovechar tanto la información espectral específica como las características espaciales de las imágenes hiperespectrales.

\vspace{5mm}

Los resultados preliminares demuestran la viabilidad del enfoque propuesto para la detección no invasiva de contaminación por micotoxinas, estableciendo un marco de trabajo que contribuirá significativamente a mejorar los estándares de seguridad alimentaria en la producción de higos, especialmente relevante para regiones productoras como Extremadura, que representa el 55.5\% de la producción nacional española.

\vspace{5mm}

\keywords{inteligencia artificial, imágenes hiperespectrales, aprendizaje automático, visión por computador, algoritmo genético, micotoxinas, aflatoxinas, seguridad alimentaria, Aspergillus flavus, higos, detección no invasiva, transformada wavelet, redes neuronales profundas, redes neuronales convolucionales}
